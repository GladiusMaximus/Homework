\item \textbf{Exercise:} Write out the powers of 2 mod 7

\(
\begin{array}[t]{ll}
2^0 \pmod 7 \equiv & 1 \\
2^1 \pmod 7 \equiv & 2 \\
2^2 \pmod 7 \equiv & 4 \\
2^3 \pmod 7 \equiv & 1 \\
2^4 \pmod 7 \equiv & 2 \\
2^5 \pmod 7 \equiv & 4 \\
2^6 \pmod 7 \equiv & 1 \\
\end{array}
\)

\item \textbf{Theorem:} Coprime numbers raised to any power are still coprime.

Let \(a, n \in \mathbb Z\) where \(\gcd(a, n) = 1\). This proof applies for all \(j \in \mathbb N\). Show \(\gcd(a^j, n) = 1\)

%TODO: Fix this typsetting
\textbf{Proof:} I will begin by using the tools developed in Chapter 2, \(\pf(a) \cap \pf(n)  = \{\}\). \(\min (x \# \pf(a), x \# \pf(b))\) \( = 0\). Since \(\min(a, b) = a \vee \min(a, b) = b\), \(0 = \pf(a) \vee 0 = \pf(b)\). If \(0 = \pf(a)\), then \(x \# \pf(a) = 0\), furthermore no matter how many a's, \(x \# \pf(a^j) = 0\) (since \(x \# \pf(a) = 0 = j (x \# \pf(a)) = x \# \pf(a^j)\)). Thus \(\min(x \# \pf(a^j), x \# \pf(b)) = 0\). Otherwise \(0 = \pf(b)\), then no matter what \(x \# \pf(a^j)\) is, \(\min(x \# \pf(a^j), x \# \pf(b)) = 0\). Thus \(\gcd(a^j, n) = 1\). This conditional proof shows \(\gcd(a, n) = 1 \rightarrow \gcd(a^j, n) = 1\). \qedhere

This proof can also be written using 1.43.

\item \textbf{Theorem:} If \(b\) is congruent to a coprime of the modulo, then \(b\) is a coprime to the modulo.

Let \(b \equiv a \pmod n\) and \(\gcd(a, n) = 1\). Show \(\gcd(a, b) = 1\)

\textbf{Proof:} Assume for contradiction \(b = n c \fs c\). Then \(b \equiv a \pmod n\) means \(n | (nc - a)\). This is problematic because then \(nj = nc - a\), and then \(n(c-j) = a\). Therefore \(b \neq nc\). Therefore by definition of greatest common divisor \(\gcd(b, n) = 1\). In conclusion \((\gcd(a, n) = 1 \wedge b \equiv a \pmod n) \rightarrow \gcd(a, b) = 1\). \qedhere

\item \textbf{Theorem:} All numbers have at least two different exponents that give the same result.

Let \(a, n \in \mathbb N\). Assume \(\neg \exists a^i \not\equiv a^j \pmod n\) for contradiction.

\textbf{Proof:} For \(i \in \{1, 2, \dots, n\}\), \(\neg \exists a^i \not\equiv a^j \pmod n\). These \(n\) noncongruent integers form a CRS by Theorem 3.17. \(a^{n+1}\) must be congruent to something in the CRS by the definition of CRS. Therefore \(\exists j (a^{n+1} \equiv a^j \pmod n)\). This can not be the case since it denies the contradictive assumption. Therefore \(\exists i, j \in \mathbb N (i \neq j \wedge a^i \equiv a^j \pmod n)\). \qedhere

\item \textbf{Theorem:} The converse of Theorem 1.14 is true if \(\gcd(c, n) = 1\).

Let \(a, b, c, n \in \mathbb N\). Let \(ac \equiv bc \pmod n\). Show \(a \equiv b \pmod n\)

\textbf{Proof:} The first congruence translates to \(n | (ac-bc)\) or \(n | c(a-b)\). By Theorem 1.41, \(n | (a-b)\) (since \(\gcd(a, n) = 1\), no factor of \(c\) can be divided by \(n\)). Therefore \(a \equiv b \pmod n\). In conclusion \(ac \equiv bc \wedge \gcd(c, n) = 1 \rightarrow a \equiv b\). \qedhere

\item \textbf{Theorem:} If a number is coprime to the modulo, it has at least one power congruent to one.

Let \(\gcd(a, n) = 1\). Show \(\exists k \in \mathbb N (a^k \equiv 1 \pmod n)\)

\textbf{Proof:} \(a^i \equiv a^j \pmod n\) Without loss of generality, \(i \geq j\). \(\frac{a^i}{a^j} \equiv \frac{a^j}{a^j} \pmod n\) by Theorem 4.5, or equivalently \(a^{i-j} \equiv a^{i-i} \equiv 1 \pmod n\). Therefore when \(k = i - j\), \(a^k \equiv 1 \pmod n\). In conclusion \(\gcd(a, n) = 1 \rightarrow \exists k \in \mathbb N (a^k \equiv 1 \pmod n)\). \qedhere

\item \textbf{Question:} Compute some orders of numbers. % TODO

\item \textbf{Theorem:} All powers of a relatively prime \(a\) up to \(\ord_n(a)\) are pair-wise incongruent modulo \(n\).

Translated: \(\gcd(a, n) = 1 \wedge i \leq \ord(a) \wedge j \leq \ord(a) \rightarrow a^i \not\equiv a^j\). All congruences and orders are taken to be mod n.

\textbf{Proof:} Assume \(a^i \equiv a^j\). Without loss of generality, \(i > j\). Then \(a^{i-j} \cdot a^j \equiv a^j \cdot 1\) which can be simplified via 4.2 and 4.5 to \(a^{i-j} \equiv 1\). But since \(\ord(a)\) is the smallest integer with this property,  \(\ord(a) \leq i - j\). Therefore \(i > \ord(a)\). \qedhere

\item \textbf{Theorem:} All powers of a relatively prime \(a\) past \(\ord_n(a)\) will never produce new numbers mod \(n\).

Translated \(i > \ord(a) \rightarrow \exists r \leq \ord(a) (a^i \equiv a^r)\). All congruences and orders are taken mod \(n\).

\textbf{Proof:} Divide \(i\) by \(\ord(a)\) such that \(i = p \cdot \ord(a) + r\) where \(0 \leq r < \ord(a)\). \(a^i = a^{p \cdot \ord(a) + r} = (a^{\ord(a)})^p \cdot a^r \equiv 1 a^r\), or \(a^i \equiv 1 \cdot a^r \equiv a^r\). Therefore \(i > \ord(a) \rightarrow \exists r \leq \ord(a) (a^i \equiv a^r)\). \qedhere

\item \textbf{Theorem:} \(a^m \equiv 1 \leftrightarrow \ord(a)|m\). All congruences and orders are taken mod \(n\).

\textbf{Proof: \(\rightarrow\)} Divide \(m\) by \(\ord(a)\) such that \(m = q \cdot \ord(a) + r\) where \(0 \leq r < \ord(a)\). \(a^m = a^{q \cdot \ord(a) + r} = (a^{\ord(a)})^q \cdot a^r \equiv 1 \cdot a^r\). \(\gcd(a^r, n) = 1\), so by Theorem 4.5 \(a^r \equiv 1\). But \(0 \leq r < \ord(a)\), so \(r = 0\). Therefore \(m|\ord(a)\).

\textbf{Proof: \(\leftarrow\)} \(\ord(a)|m\) implies \(j \cdot \ord(a) = m\). \(a^m = a^{j \cdot \ord(a)} = (a^{\ord(a)})^j \equiv 1^j = 1\).

In conclusion \(\ord(a)|m \leftrightarrow a^m \equiv 1\). \qedhere

\item \textbf{Theorem:} The order of a coprime is less than the modulo.

Translated: \(\gcd(a, n) = 1 \rightarrow \ord(a) < n\). All orders and congruences are taken mod \(n\).

\textbf{Proof:} There can not be more than \(n\) unique numbers modulo \(n\) by Theorem 3.16. \(a^i\) for \(0 \leq i < \ord(a)\) produces unique numbers modulo \(n\). Therefore there \(\ord(a) < n\). \qedhere

\item \textbf{Exercise:} Compute the following expression for several natural numbers \(a\) and prime numbers \(p\) \(a^{p-1} \pmod p\).

I conjecture that \(\ord(a) < n\).

\usemintedstyle{colorful}
\begin{minted}[mathescape,linenos]{python}
def mod_exp(a1, r, n):
    # Returns the k in $a^r \equiv k \pmod{n}$ where $0 \leq k < r$
    # This algorithm is found in 3.6
    # WLOG a < n
    a = cmod(a1, n) # reduce a mod n if possible
    a_squared = cmod(a * a, n)
    r_halved, remainder = division(r, 2)
    if r == 1:
        # Base case
        return a
    if divides(2, r):
        # $(a^2)^{r/2}$
        k = mod_exp(a_squared, r_halved, n)
        k = cmod(k, n) # reduce k mod n
        return k
    else:
        # $(a^2)^{(r-1)/2} \cdot a$
        k = mod_exp(a_squared, r_halved, n)
        ka = cmod(k * a, n)
        return ka

for p in first(10, primes()):
    print(r'\(\pmod {{{p}}}\)'.format(**locals()))
    print('')
    print(r'\begin{tabular}[t]{l}')
    for a in range(0, p):
        # $ 0 \leq a < p$
        e = p - 1
        c = mod_exp(a, e, p, False)
        print(r'${a}^{{{e}}} \equiv {c} \pmod {{{p}}}$ \\'.format(**locals()))
    print(r'\end{tabular}')
    print('')
\end{minted}

Output:

\(\pmod {2}\)

\begin{tabular}[t]{l}
$0^{1} \equiv 0 \pmod {2}$ \\
$1^{1} \equiv 1 \pmod {2}$ \\
\end{tabular}

\(\pmod {3}\)

\begin{tabular}[t]{l}
$0^{2} \equiv 0 \pmod {3}$ \\
$1^{2} \equiv 1 \pmod {3}$ \\
$2^{2} \equiv 1 \pmod {3}$ \\
\end{tabular}

\(\pmod {5}\)

\begin{tabular}[t]{l}
$0^{4} \equiv 0 \pmod {5}$ \\
$1^{4} \equiv 1 \pmod {5}$ \\
$2^{4} \equiv 1 \pmod {5}$ \\
$3^{4} \equiv 1 \pmod {5}$ \\
$4^{4} \equiv 1 \pmod {5}$ \\
\end{tabular}

\(\vdots\)

Output has been omitted for brevity.

\(\vdots\)

\begin{tabular}[t]{l}
$11^{22} \equiv 1 \pmod {23}$ \\
$12^{22} \equiv 1 \pmod {23}$ \\
$13^{22} \equiv 1 \pmod {23}$ \\
$14^{22} \equiv 1 \pmod {23}$ \\
$15^{22} \equiv 1 \pmod {23}$ \\
$16^{22} \equiv 1 \pmod {23}$ \\
$17^{22} \equiv 1 \pmod {23}$ \\
$18^{22} \equiv 1 \pmod {23}$ \\
$19^{22} \equiv 1 \pmod {23}$ \\
$20^{22} \equiv 1 \pmod {23}$ \\
$21^{22} \equiv 1 \pmod {23}$ \\
$22^{22} \equiv 1 \pmod {23}$ \\
\end{tabular}

What I find interesting is that this program builds off of the one from 3.6. The tools I develop build off of each other. That is the whole idea behind reusable functions in a programming language.

\item \textbf{Theorem:} Let \(S = \{a, 2a, 3a, \dots, pa\}\) where \(\gcd(a, p) = 1\). \(S\) is a complete residue system modulo \(p\).

\textbf{Proof:} Let \(R = \{1, 2, 3, \dots, p\}\). \(R\) is the canonical complete residue system modulo \(p\). Therefore all elements of \(R\) are pairwise incongruent \(\forall i, j (i \neq j \rightarrow i \not\equiv j \pmod p)\). The contrapositive of theorem 4.5 states that \(i \not\equiv j \pmod n\) implies \(ai \not\equiv aj \pmod n\). Therefore the elements of \(R\) are also pairwise incongruent. By Theorem 3.17, any set of \(p\) pairwise incongruent integers form a complete residue system modulo \(p\).

\item \textbf{Theorem:} Let \(p \in \mathbb P\) and \(a \nmid p\). \(a \cdot 2a \cdot 3a \cdot \dots \cdot (p-1)a \equiv 1 \cdot 2 \cdot 3 \cdot \dots \cdot (p-1) \pmod p\).

Let \(R = \{a, 2a, 3a, \dots, (p-1)a, pa\}\) and \(S = \{1, 2, 3, \dots, p-1, p\}\).

\textbf{Proof:} \(R\) consitutes a complete residue system by Theorem 3.14. \(S\) is the canonical complete residue system. Therefore every element of \(R\) is congruent to exactly one thing in \(S\) and everything in \(S\) is congruent to exactly one thing in \(R\). Therefore there is a one-to-one mapping of congruent elements from \(S\) to \(R\). \(p|pa \wedge p|p\), therefore \(pa \equiv 0 \equiv p \pmod p\). Therefore there is a one-to-one mappnig of congruent elements from \(R \setminus \{pa\}\) to \(S \setminus \{p\}\). For each element pair \(r_i\) and \(s_i\) in \(R \setminus \{pa\}\) and \(S \setminus \{p\}\), we can multiply the left-hand side of the equation \(1 \equiv 1 \pmod p\) by \(r_i\) and the right-hand side by \(s_i\). In the end, we will get all of the elements of \(R \setminus \{pa\}\) multiplied together are equivalent to all of the elements of \(S \setminus \{p\}\). \(a \cdot 2a \cdot 3a \cdot \dots \cdot (p-1)a \equiv 1 \cdot 2 \cdot 3 \cdot \dots \cdot (p-1) \pmod p\). \qedhere

\item \textbf{Theorem:} (Fermat's Little Theorem) In a prime modulo, an integer not divisible by the modulo raised to the \((p-1)\)-th power is congruent to one.

Let \(p \in \mathbb P\) and \(a \in \mathbb Z \wedge p \nmid a\). \(a^{p-1} \equiv 1 \pmod p\).

\textbf{Proof:} \(a \cdot 2a \cdot 3a \cdot \dots \cdot (p-1)a \equiv 1 \cdot 2 \cdot 3 \cdot \dots \cdot (p-1) \pmod n\) by Theorem 4.13. Then \(1 \cdot 2 \cdot 3 \cdot \dots \cdot (p-1) \cdot a^{p-1} \equiv 1 \cdot 2 \cdot 3 \cdot \dots \cdot (p-1) \pmod p\). Since \(p \in \mathbb P\), \(\forall i < p (\gcd(i, p) = 1)\), we can repeatedly apply Theorem 4.14. Therefore \(a^{p-1} \equiv 1 \pmod n\). \qedhere

\item \textbf{Theorem:} (Fermat's Little Theorem) In a prime modulo, an integer raised to the power of the modulo is congruent to itself.

Let \(p \in \mathbb P\) and \(a \in \mathbb Z\). \(a^p \equiv a \pmod p\)

\textbf{Proof:} Let \(p \in \mathbb P\) and \(a \in \mathbb Z\). \(a^{p-1} \equiv 1 \pmod p\) by Theorem 4.15. If \(p \nmid a\), then \(a^{p-1} \equiv 1 \pmod p\), therefore \(a^p \equiv a \pmod p\) by Theorem 1.14. On the other hand, if \(p | a\), then \(a \equiv 0 \equiv a^p \pmod p\). Therefore \(a^p \equiv a \pmod p\) in both cases.

\item \textbf{Note:} 4.15 and 4.16 are equivalent.

\textbf{Proof that 4.15 implies 4.16:} See proof of 4.16 (which relies on 4.15).

\textbf{Proof that 4.16 implies 4.15:} Let \(p \in \mathbb P\) and \(a \in \mathbb Z \wedge p \nmid a\). \(a^p \equiv a \pmod p\) by Theorem 4.16. Since \(p \in \mathbb P\) and \(p \nmid a\), \(\gcd(a, p) = 1\). This lets us apply Theorem 4.5 to the equation \(a^{p-1} a \equiv 1 a \pmod p\), yielding \(a^{p-1} \equiv 1 \pmod p\).

\item In a prime modulo, one less than the modulo divides the order of an integer coprime to that modulo.

Let \(p \in \mathbb P\) and \(a \in \mathbb Z \wedge p \nmid a\). \(\ord_p(a) | (p-1)\)

\textbf{Proof:} \(a^{p-1} \equiv 1 \pmod p\) by Theorem 4.15. \((p-1)|\ord_p(a)\) by Theorem 4.10.

\item \textbf{Exercise:} Use Fermat's Little Theorem to efficiently raise numbers to large powers in modulo arithmetic.

\begin{enumerate}
\item \(512^{372} = 512^{31 \cdot 12} = (512^{12})^{31} \equiv 1^{31} \pmod{13} = 1\)
\item \(3444^{3233} = 3444^{202 \cdot 16 + 1} = (344^{16})^{212} \cdot 344^1 \equiv 1^{202} \cdot 344 \pmod{17} = 344\)
\item \(123^{456} \equiv (2^3)^{456} \pmod{23} = 2^{3 \cdot 456} = 2^{62 \cdot 22 + 4} = (2^{22})^{62} \cdot 2^4 \equiv 1^{62} \cdot 2^4 \pmod{23} = 16\)
\end{enumerate}

\item \textbf{Exercise:} Find the remainder upon division of \(314^{159}\) by \(31\)

\(314^{159} \equiv (2^2)^{159} \pmod {31} = 2^{2 \cdot 159} = 2^{5 \cdot 62 + 3} = (2^5)^{62} \cdot 2^3 \equiv 1^62 \cdot 2^3 \pmod {31} = 8\)

The remainder upon division is 8.\(2^{144} = (2^{12})^{12} \equiv 1^{12} \pmod 13 = 1\).

\item \textbf{Theorem:} \(x \equiv a \pmod n\), \(x \equiv a \pmod m\), and \(\gcd(n, m) = 1\) imply \(x \equiv a \pmod {mn}\).

\textbf{Proof:} \(n|(x-a)\) and \(m|(x-a)\). \(mn|(x-a)\) by Theorem 2.25. \qedhere

\item \textbf{Exercise:} The remainder of \(4^{72}\) divided by \(91\) is 8.

\(2^{144} = (2^{12})^{12} \equiv 1^{12} \pmod {13} = 1\). Therefore \(x \equiv 1 \pmod {13}\)

\(2^{144} = (2^{2})^{72} \equiv 1^{72} \pmod 3 = 1\). Therefore \(x \equiv 1 \pmod 3\).

Therefore \(4^{72} = 2^{144} \equiv 1 \pmod {91}\) by Theorem 4.21.

% My class skipped a bunch here

\setcounter{enumii}{27}

\item \textbf{Theorem:} \(\gcd(a, b) = 1 \wedge \gcd(a, c) = 1 \rightarrow \gcd(a, bc) = 1\)

\textbf{Proof:}

\begin{tabular}[t]{ll}
\multicolumn{2}{l}{Let \(\pf(a) = A\), \(\pf(b) = B\), \(\pf(c) = C\)} \\
\(A \cap B = \{\}\) \\ j 
\(A \cap C = \{\}\) & Coprime-disjoint theorem \\
\(\gcd(a, bc) = A \cap (\pf(bc))\) & GCD-intersection theorem \\
\(\quad = A \cap (B + C)\) & pf of product \\
\(\quad = A \cap B + A \cap C\) & Empty-intersection theorem \\
\(\quad = \{\} + \{\}\) & Substitution \\
\(\quad = \{\}\) & Identity \\
\(\gcd(a, bc) = 1\) & Coprime-disjoint theorem \qedhere
\end{tabular}

\item \textbf{Theorem:} Let \(b \equiv a \pmod n\) and \(\gcd(a, n) = 1\). Show \(\gcd(a, b) = 1\)

\textbf{Proof:} Assume for contradiction \(b = n c \fs c\). Then \(b \equiv a \pmod n\) means \(n | (nc - a)\). This is problematic because then \(nj = nc - a\), and then \(n(c-j) = a\), and then \(n|a\), and then \(\gcd(a, n) = n\). Therefore \(b \neq nc\). Therefore by definition of greatest common divisor \(\gcd(b, n) = 1\). In conclusion \((\gcd(a, n) = 1 \wedge b \equiv a \pmod n) \rightarrow \gcd(a, b) = 1\). \qedhere

\item \textbf{Theorem:} Let \(a, b, c, n \in \mathbb N\). Let \(ac \equiv bc \pmod n\). Show \(a \equiv b \pmod n\)

\textbf{Proof:} The first congruence translates to \(n | (ac-bc)\) or \(n | c(a-b)\). By Theorem 1.41, \(n | (a-b)\) (since \(\gcd(a, n) = 1\), no factor of \(c\) can be divided by \(n\)). Therefore \(a \equiv b \pmod n\). \qedhere

\item \textbf{Theorem:} Let \(x_1, x_2, \dots, x_{\phi(n)}\) be the natural numbers relatively prime to \(n\) and less than \(n\). Let \(\gcd(a, n) = 1\) (but not necessarily \(a \leq n\), so not necessarily \(\exists i (a = x_i)\)). \(i \neq j \rightarrow ax_i \not\equiv ax_j\)

All congruences are taken modulo \(n\).

\textbf{Proof:} \(ax_i \equiv ax_j\) implies \(x_i \equiv x_j\) by Theorem 4.30, or equivalently \(n|(x_i - x_j\). Since \(0 \leq x_j < n\) and without loss of generality \(x_j \leq x_i < n\), \(0 \leq x_i - x_j < n\), but \(n|(x_i - x_j)\), therefore \(x_i - x_j = 0\). Therefore \(x_i = x_j\). This contradicts. Therefore \(ax_i \not\equiv ax_j\). \qedhere

\item \textbf{Theorem:} (Euler's Theorem) \(a^{\phi(n)} \equiv 1 \pmod n\)

By Theorem 4.31, the members of the set \(\{ax_1, ax_2, \dots, x_{\phi(n)}\}\) are pairwise incongruent.

\item \textbf{Theorem:} (Fermat's Little Theorem) \(a^{(p-1)} \equiv 1 \pmod n\).

\textbf{Proof:} If \(n \in \mathbb P\), then all natural numbers less than \(n\) are coprime to \(n\). Therefore \(\phi(n)\) counts all numbers from 1 to \(n-1\). Therefore \(\phi(n) = n - 1\). Therefore \(a^{(p-1)} \equiv 1 \pmod n\). \qedhere

\item \textbf{Exercise:}
\begin{enumerate}
\item \(4^{49} \equiv 12^{49} \equiv ? \pmod{15}\)
\item \(139^{112} \equiv ? \pmod {27}\)
\end{enumerate}

\item \textbf{Exercise:} Find the ones digit of \(13^{474}\)

\(13^{174} = (13^4)^{18} \cdot 13^2 \equiv 1^{18} \cdot 3^2 \pmod {10} = 9\)

\item \textbf{Theorem:} Every number has a multiplicative inverse in a prime modulo.

\textbf{Proof:} By Fermat's Little Theorem \(a^{p-1} \equiv 1\). Since \(p \geq 2\), \(a^{p-2} a = a^{p-1} \equiv 1\). Therefore reduce \(a^{p-2}\) into the CCRS where \(a^{p-2} \equiv b\). \(\forall 1 < a < p - 1 \exists 1 < b < p - 1 ab \equiv 1\). \qedhere

\item \textbf{Theorem:} \(1\) and \(p-1\) are their own multiplicative inverses in a prime modulo \(p\).

Translated: \(1 \cdot 1 \equiv 1\) and \((p-1) \cdot (p-1) \equiv 1\). All congruences are taken mod \(p\)

\textbf{Proof:} \(1 \cdot 1 = 1 \equiv 1\). \((p-1)\cdot(p-1) = p^2 - 2p + 1 = (p-2) \cdot p + 1 \equiv 1\). \qedhere

\item \textbf{Theorem:} No other number (besides \(1\) and \(p-1\)) is its own inverse in a prime modulo \(p\).

Translated: \(0 \leq a < p \wedge a^2 \equiv 1 \rightarrow a \equiv 1 \vee a \equiv p-1\), where all congruences are taken in a prime modulo \(p\).

\textbf{Proof:} Let \(a^2 \equiv 1\). By the definition of modulo, \(p|(a^2 - 1)\), or equivalently \(p|(a-1)(a+1)\). \(p \in \mathbb P\), therefore \(\gcd(p, a-1) = 1\) and \(\gcd(p, a+1) = 1\) (unless \(a+1\) was \(p\) or \(a-1\) was \(0\)). By Theorem 4.28, \(\gcd(p, (a-1)(a+1)) = 1\). Therefore \(p\nmid (a-1)(a+1\) unless \(a = p - 1\) or \(a = 1\). But we know that \(p|(a-1)(a+1)\) from the premise, so \(a = p-1\) or \(a = 0\). \qedhere

%\setcounter{enumi}{6}
%\setcounter{enumii}{3}
%\item \textbf{Theorem:} 

% I'm an idealist. I don't know where I'm going, but I'm on my way. Carl Sandburg
% It takes considerable knowledge just to realize the extent of your own ignorance. Thomas Sowell
% 'But I don't want to go among mad people,' Alice remarked. 'Oh, you can't help that,' said the Cat: 'we're all mad here. I'm mad. You're mad.'
% Don't Panic. Douglas Adams
% This planet has - or rather had - a problem, which was this: most of the people living on it were unhappy for pretty much of the time. Many solutions were suggested for this problem, but most of these were largely concerned with the movement of small green pieces of paper, which was odd because on the whole it wasn't the small green pieces of paper that were unhappy. Douglas Adams.
% So long, and thanks for all the fish. Douglas Adams
% man had always assumed that he was more intelligent than dolphins because he had achieved so much—the wheel, New York, wars and so on—whilst all the dolphins had ever done was muck about in the water having a good time. But conversely, the dolphins had always believed that they were far more intelligent than man—for precisely the same reasons. Douglas Adams

%%% Local Variables:
%%% mode: latex
%%% TeX-master: "main"
%%% End:
