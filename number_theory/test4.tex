\documentclass[12pt,letterpaper,english]{article}
\usepackage[utf8]{inputenc}
\usepackage[T1]{fontenc}
\usepackage[english]{babel}
\usepackage{amsmath}
\usepackage{amssymb}
\usepackage[letterpaper,margin=1in]{geometry}
\usepackage{setspace}
\usepackage[pdftex]{graphicx}
\usepackage{fancyhdr}
\usepackage{lastpage}
\usepackage{minted}
\usepackage{cancel}

\pagestyle{fancy}
\fancyhead{}
\fancyfoot{}
\fancyhead[HR]{\thepage\ of \pageref{LastPage}}

\DeclareMathOperator{\ord}{ord}
\newcommand{\qedhere}{{\tiny \(\blacksquare\)}}

\begin{document}

\doublespacing
\begin{center}
{\Large Test 2} \\[14pt]
{\large Sam Grayson} \\[0pt]
{\today} \\
\end{center}

\singlespacing
\setlength{\parindent}{0pt}

\begin{enumerate}
%[leftmargin=0mm]

\item \(p \in \mathbb P \wedge q \in \mathbb P \rightarrow p^{q-1} + q^{p-1} \equiv 1 \pmod {pq}\).

\textbf{Proof:} \(p^{q-1} \equiv 1 \pmod q\) and \(q^{p-1} \equiv 1 \pmod p\) by Fermat's Little Theorem. \(q^{p-1} \equiv 0 \pmod q\) and \(p^{q-1} \equiv 0 \pmod p\) (since always \(a|a^i\)). Then \(p^{q-1} + q^{p-1} \equiv 1 + 0 \equiv 1 \pmod q\) and \(p^{q-1} + q^{p-1} \equiv 0 + 1 \equiv 1 \pmod q\).  By Theorem 4.21, \(p^{q-1} + q^{p-1} \equiv 1 \pmod {pq}\). \qedhere

\item Let \(i \in \mathbb Z\) where \(\gcd(\ord(a), i) = 1\). Then \(\ord(a^i) = \ord(a)\).

\textbf{Proof:} \((a^{\ord(a)})^i \equiv 1 \equiv (a^i)^{\ord(a)}\), so \(\ord(a^i)|\ord(a)\). In a similar way, \((a^i)^{\ord(a^i)} \equiv 1 \equiv a^{i \cdot \ord(a^i)}\), so \(\ord(a)|(i \ord(a^i))\). But by Theorem 1.39, since \(\gcd(\ord(a^i), i) = 1\), \(\ord(a)|\ord(a^i)\). Together with \(\ord(a^i)|\ord(a)\), this proves that \(\ord(a^i) = \ord(a)\). \qedhere

\item Let \(\gcd(a, m) = 1\) for \(a, m \in \mathbb Z\). \(\ord(a)|\phi(m)\). (All orders and congruences are taken mod \(m\).)

\textbf{Proof:} \(a^{\phi(m)} \equiv 1\) by Theorem 4.32 (Euler's Theorem). \(\ord(a)|\phi(m)\) by Theorem 4.10. \qedhere

\item \(\phi(pq) = \phi(p) \phi(q)\). This is not necessarily true when \(p = q\).

\textbf{Counter example:} Consider the case where \(p = 5\) and \(q = 5\).

\begin{tabular}[t]{l}
\(\phi(5) = 4 \quad \{1, 2, 3, 4, \cancel{5}\}\) \\
\(\phi(25) = 20 \quad \{1, 2, 3, 4, \cancel{5}, 6, 7, 8, 9, \cancel{10}, 11, 12, 13, 14, \cancel{15}, 16, 17, 18, 19, \cancel{20}, 21, 22, 23, 24, \cancel{25}\}\) \\
\(\phi(5) \cdot \phi(5) = 4 \cdot 4 = 16 \neq \phi(25)\) \\
\end{tabular}

\item 

\item \textbf{Code:}

\usemintedstyle{colorful}
\begin{minted}[mathescape,linenos]{python}
def gcd(a1, b1):
    # Returns the greatest common multiple
    # WLOG $a > b > 0$
    a = max(abs(a1), abs(b1))
    b = min(abs(a1), abs(b1))
	# find the remainder upon division
    q, r = division(a, b)
    if r == 0:
        return b
    else:
        return gcd(b, r)

def coprime(a, b):
    # Returns True if a and b are coprime
    return gcd(a, b) == 1

def phi(n):
    count = 0
    for i in range(1, n+1): # $1 \leq i < n+1$
        if coprime(i, n):
            count = count + 1
    return count

for x in range(10000):
    if phi(x) == 24:
        print(x)
\end{minted}

\textbf{Output:}

\(\phi(x) = 24 \leftrightarrow x \in \{35, 39, 45, 52, 56, 70, 72, 78, 84, 90\}\)

\item I really enjoyed your course. This was a highlight of my A-day schedule.

\end{enumerate}

\end{document}

%%% Local Variables:
%%% mode: latex
%%% TeX-master: t
%%% End:
