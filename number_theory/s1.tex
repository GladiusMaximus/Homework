\item 
\begin{proof}
$ma = b \fs m$ & Definition of `divides' \\
$na = c$ \fs n& Definition of `divides' \\
$na + ma = b + c$ & Algebra \\
$(n + m) a = b + c$ & Algebra \\
$a | (b + c)$ & Definition of `divides'
\end{proof}

\item 
\begin{proof}
Let $d = -c$ \\
$a | (b + d)$ & Theorem 1.1 \\
$a | (b - c)$ & substitution
\end{proof}

\item 
\begin{proof}
$ma = b \fs m$ & Definition of `divides' \\
$na = c \fs n$ & Definition of `divides' \\
$mana = b c$ & Algebra \\
$a | bc$ & Definition of `divides'
\end{proof}

\item 
\begin{proof}
$mana = b c$ & see last proof \\
$a^2 | bc$ & Definition of `divides'
\end{proof}

\item 
\textit{If $a|b$ then $a|b^n$}

\begin{proof}
$b = ka \fs k$ & Definition of `divides' \\
$b^n = (ka)^n = k k^{(n-1)} a^n$ & Algebra \\
$k | b^n$ & Definition of `divides'
\end{proof}

\item 
\begin{proof}
$ka = b \fs k$ & Definition of `divides' \\
$ack = bc$ & Algebra \\
$a|bc$ & Definition of `divides'
\end{proof}

\item 
\begin{enumerate}
\item $45 - 9 = 36 = 9 \cdot 4$. True
\item $37-2=35=7\cdot5$. True
\item $37-3=34$. False
\item $37 - (-3) = 40 = 8\cdot5$. True
\end{enumerate}

\item 
\begin{proof}
let $k$ be all the numbers \\
where $k \equiv b \pmod 3$ \\
$3 | (k - b)$ & Definition of `mod' \\
$3n = k - b \fs n$ & Definition of `divides' \\
$3n + k = n$ & Algebra
\end{proof}
\begin{enumerate}
\item $3n$
\item $3n + 1$
\item $3n + 2$
\item $3n$
\item $3n + 1$
\end{enumerate}

\item 
\begin{proof}
$a - a = 0 = 0n$ & Arithmetic \\
$n | (a - a)$ & Definition of `divides' \\
$a \equiv 0 \pmod n$ & Definition of `mod'
\end{proof}

\item 
\begin{proof}
$n | (a - b)$ & Definition of `mod' \\
$kn = a - b \fs k$ & Definition of `divides' \\
$-kn = b - a$ & Algebra \\
$n|(b - a)$ & Definition of `divides' \\
$b \equiv a \pmod n$
\end{proof}

\item 
\begin{proof}
$n | (a - b)$ & Definition of `mod' \\
$n | (b - c)$ & Definition of `mod' \\
$n | (a - b + b - c)$ & Theorem 1.1 \\
$n | (a - c)$ & Algebra \\
$a \equiv c \pmod n$ & Definition of `mod'
\end{proof}

\item 
\begin{proof}
$n|(a - b)$ & Definition of `mod' \\
$n|(c - d)$ & Definition of `mod' \\
$n|(a + c - b - d))$ & Theorem 1.1 \\
$n|((a + c) - (b + d))$ & Algebra \\
$a + c \equiv b + d \pmod n$ & definion `mod'
\end{proof}

\item 
\begin{proof}
let $e = -c$ and $f = -d$ \\
$a + e \equiv b + f$ & Theorem 1.12 \\
$a - c \equiv b - d$ & substitution
\end{proof}

\item 
\begin{proof}
$n | (a - b)$ & Definition of `mod' \\
$n | (c - d)$ & Definition of `mod' \\
$n | (a - b)(c - d)$ & Theorem 1.3
\end{proof}

\item 
\begin{proof}
$a \equiv b \pmod n$ & Premise \\
$a^2 \equiv b^2 \pmod n$ & Theorem 1.14
\end{proof}

\item 
\begin{proof}
$a \equiv b \pmod n$ & Premise \\
$a^2 \equiv b^2 \pmod n$ & Theorem 1.15 \\
$a^2 a \equiv b^2 b \pmod n$ & Theorem 1.14 \\
$a^3 \equiv b^3 \pmod n$ & Algebra
\end{proof}

\item 
\begin{proof}
$a \equiv b \pmod n$ & Premise \\
$a^{k - 1} \equiv b^{k - 1} \pmod n$ & Premise \\
$a^{k - 1} a \equiv b^{k - 1} b \pmod n$ & Theorem 1.14 \\
$a^k \equiv b^k \pmod n$ & Algebra
\end{proof}

\item 
\begin{proof}
Base case: \\
$a \equiv b \pmod n$ & Premise \\
Inductive Hypothesis: \\
$a^{k - 1} \equiv b^{k - 1} \pmod n$ & (assumption) \\
Inductive step: \\
$a^{k - 1} a \equiv b^{k - 1} b \pmod n$ & Theorem 1.14 \\
$a^k \equiv b^k \pmod n$ & Algebra \\
Conclusion: \\
$a^k \equiv b^k \pmod n$ & inductively
\end{proof}

\item 
\begin{enumerate}
\setcounter{enumiii}{11}
\item
$6 \equiv 2 \pmod 4$ \\
$5 \equiv 1 \pmod 4$ \\
$6 + 5 \equiv 2 + 1 \pmod 4$ \\
\item 
$6 - 5 \equiv 2 - 1 \pmod 4$ \\
\item 
$6 \cdot 5 \equiv 2 \cdot 1$ \\
\item 
$6^2 \equiv 2^2 \pmod 4$ \\
\item 
$6^3 \equiv 2^3 \pmod 4$ \\
\item 
$6^4 \equiv 2^4 \pmod 4$ \\
\item 
$6^k \equiv 2^k \pmod 4$ \\
\end{enumerate}

\item No \\
Consider the case wehre $n = 4$, $c = 0$, $a = 1$, and $b = 2$. \\
$ac \equiv bc \pmod n$ \\
$a \neq b$ \\

\item See 1.22 and 1.23

\item 
\begin{proof}
$3|a$ & Premise (Base Case) \\
$3|b$ & Let $b$ be an integer \ldots (Inductive Hypothesis) \\
$3|9$ & Arithmetic \\
$3|(9  b_k  10^{k - 1})$ & Theorem 1.3 \\
$3|(b - 9  b_k  10^{k - 1})$ & Theorem 1.2 \\
$3|(b_{k - 1} + b_{k})b_{k - 2} \ldots b_0$ & Algebra* (Inductive Step)\\
$3|(a_{k} + a_{k - 1} + a_{k - 2} + \ldots a_1 + a_0)$ & Inductive axiom
\end{proof}

Here is the algebra I used in the step labeled `Algebra*':

\(
\begin{array}{rc}
b - b_k 9 10^{k - 1} & = \\
b - b_k (10 - 1) 10^{k - 1} & = \\
b + (-b_k  10 \cdot 10^{k - 1} + b_k  1  10^{k - 1}) & = \\
b + (-b_k  10^k + b_k  10^{k - 1}) & = \\
\begin{array}{r ccccc l}
& b_k & b_{k - 1} & b_{k - 2} & \ldots & b_0 \\
+ \quad & (-b_k) & b_k & 0 &\ldots & 0 & \quad = \\
\hline
& & (b_k + b_{k - 1}) & b_{k - 2} & \ldots & b_0 \\
\end{array}
\end{array}
\)

\item
\begin{proof}
$3|a$ & Premise (Base Case) \\
$3|(b_k + b_{k - 1} + \ldots + b_0)$ & Assumption (Inductive Hypothesis) \\
$3|9$ & Arithmetic \\
$3|(b_k 9  c)$ where c is $k$ ones in a row & Theorem 1.3 \\
$3|(b_k + b_{k - 1} + \ldots + b_0 + b_k  9  c)$ & Theorem 1.2 \\
$3|(b_k  10^k + b_{k - 1} + \ldots + b_0)$ & Algebra* \\
$3|(a_k 10^k + a_{k - 1} 10^{k - 1} + \ldots + a_0 10^0)$ & Inductive Axiom \\
$3|(a_k a_{k - 1} \ldots a_0)$ & Definition of digits
\end{proof}

Here is the algebra I used in the step labeled `Algebra*':

\(
\begin{array}{rcl}
b_k + b_{k - 1} + \ldots + b_0 + b_k  9  c & = \\
b_k + b_{k - 1} + \ldots + b_0 + b_k d & = & \textrm{where d is a number with $k$ nines} \\
b_k + b_{k - 1} + \ldots + b_0 + b_k (10^k - 1) & = \\
b_k + b_{k - 1} + \ldots + b_0 + b_k 10^k - b_k & = \\
b_{k - 1} + \ldots + b_0 + b_k 10^k
\end{array}
\)

\item
$4|a$ if and only if $4|(a_1 + a_3 + \ldots)(a_0 + a_2 + a_4 + \ldots)$

\item
\begin{enumerate}
\item $m = nq + r$ where $m = 25$, $n = 7$, $q = 3$, and $r = 4$
\item $m = 277$, $n = 4$, $q = 66$, and $r = 1$
\item $m = 33$, $n = 11$, $q = 3$, $r = 0$
\item $m = 33$, $n = 45$, $q = 0$, $r = 33$
\end{enumerate}

\item
\begin{tabular}[t]{p{4 in} l}
Setup: \\
\textit{Make a list of multiples of $n$ that are greater than $m$ and choose the smallest one to define $n(q + 1)$.} \\
$A := \{k | k \in \mathbb{N}\ \wedge kn > m \}$ &  \\
$\exists a \ni (a \in A \wedge an > m \wedge \forall k \in A (a \leq k))$ & Well-ordering Principle \\
$q : = a - 1 $ &  \\
$r := m - nq$ &  \\
\\
Proving $r$ satisfies upper bound: \\
\textit{If it didn't, then $a$ wouldn't be an element of $A$, but we know that $a$ is in $A$.} \\
$r > n - 1$ & Assume for contradiction \\
$r \geq n$ & Property of inequalities (over $\mathbb{Z}$) \\
$\exists j \ni (r - n = j \wedge j \geq 0)$ & Property of inequalities \\
$nq + r = m $ & Algebra (from definition of $r$) \\
$nq + (n + j) = m$ & Algebra (from definition of $j$)\\
$n(q + 1) + j = m$ & Algebra \\
$n(q + 1) \leq m$ & Property of inequalities \\
$n(q + 1) > m$ & Algebra (from definition of $a$) \\
$\therefore r \leq n - 1$ & Contradiction \\
\\
Proving $r$ satisfies lower bound: \\
\textit{If it didn't, then there would be another element smaller than $a$ in $A$, but $a$ is the least element in $A$.} \\
$r < 0$ & Assume for contradiction \\
$nq + r = m $ & Algebra (from definition of $r$) \\
$nq > m$ & Property of inequalities \\
$q \in A$ & $q \in \mathbb{N} \wedge nq > m$ is the condition for $A$ \\
$\forall k (k \in A \rightarrow q + 1 \leq k)$ & Definition of $a$ (smallest element in $A$)\\
$q + 1 \leq q$ & Universal instantiation \\
$\therefore r \geq 0$ & Contradiction \\
\\
Proving $q$ and $r$ are integers: \\
\textit{They all came from sets that only contain integers.} \\
$A \subset \mathbb{N} \subset \mathbb{Z}$ & Stuff I learned \\
$a \in A$ & Definition of $a$ \\
$a \in \mathbb{Z}$ & Property of sets \\
$q \in \mathbb{Z}$ & Closure (Definition of $q$) \\
$r \in \mathbb{Z}$ & Closure (definition of $r$) \\
\\
\end{tabular}

\item 
\begin{tabular}[t]{p{3.5in} p{3in}}
$\exists q', r' \in \mathbb{Z} (m = q'n + r' \wedge r' \neq r \wedge q' \neq q \wedge 0 \leq r \leq q' - 1)$ & Assume for contradiction \\
$r' < n$ & Assumption (restriction on $r'$) \\
$q'n + n > m$ & Property of inequalities (because $q'n + r = m$)\\
$n(q' + 1) > m$ & Algebra \\
$q' + 1 \in A$ & Definition of $A$ \\
$q' + 1 \neq q + 1$ & Property of inequalities \\
$q' + 1 > q + 1$ & Definition of $a$ (smallest element in $A$) \\
$q' \geq q + 1$ & Property of inequalities (over $\mathbb{Z}$) \\
$qn + r = m$ & Definition of $r$ \\
$qn + n > m$ & Property of inequalities (replace $r$ with something greater-than $r$) \\
$(q + 1)n > m$ & Algebra \\
$q'n > m$ & Property of inequalities (replace $q + 1$ with something greater-than-or-equal to it) \\
$q'n + r' > m$ & Property of inequalities (add a positive number to the bigger side and it is still bigger) \\
$\neg \exists q', r' \in \mathbb{Z} (m = q'n + r' \wedge r' \neq r \wedge q' \neq q \wedge 0 \leq r \leq q' - 1)$ & Contradiction \\
\end{tabular}

\item 
\begin{proof}
$n|(a - b)$ & Definition of modulo \\
$a - b = cn \fs c$ & Definition of divides \\
$b = dn + e \wedge 0 \leq e \leq n - 1$ & Division algorithm \\
$a - dn - e = cn$ & Algebra \\
$a = (c + d)n + e \wedge 0 \leq e \leq n - 1$ & Algebra \\
This satisfies the division algorithm \\
$(c + d)n + e - b = cn$ & Algebra \\
$b = dn + e \wedge 0 \leq e \leq n - 1$ & Algebra \\
Therefore, same remainder (namely $e$) & 
\end{proof}

\begin{proof}
$a = cn + r$ & Let $r$\\
$b = dn + r$ & Let $r$\\
$a - b = cn - dn = (c - d)n$ & Algebra \\
$n|(a - b)$ & Definition of divides \\
\end{proof}

\item  Yes. $1$

\item No. There are a finite number of integer factors.

\item
\begin{enumerate}
\item No
\item No
\item No
\item Yes
\item Yes
\item Yes
\end{enumerate}

\item
\begin{proof}
$a - nb = r$ & Algebra (from premise)\\
$k|nb$ & Theorem 1.3 \\
$k|(a - nb)$ & Theorem 1.2 \\
$k|r$ & Subsitution
\end{proof}

\item 
Lemma: Let $a = n b + r$. $k|b$ and $k|r$ imply $k|a$.

\begin{proof}
$k|nb$ & Therorem 1.3 \\
$k|(nb + r)$ & Theorem 1.1 \\
$k|a$ & Substitution
\end{proof}

\begin{tabular}[t]{p{4 in} l}
$(a, b) = k$ & Let \\
$k|a$ & Definition of $k$ (GCD) \\
$k|b$ & Definition of $k$ (GCD) \\
$k|r_1$ & Theorem 1.32 \\
\hline
\textit{At this point, we know that $k$ is a common divisor. Assume for the sake of contradiction that $k$ is not the greatest common divisor.} \\
$(b, r_1) = m \wedge m > k$ & Assume for contradiction \\
$m|a$ & Lemma \\
$m|b$ & Definition of GCD \\
$(b, r_1) > m \wedge m > k$ & Definition of GCD \\
$(b, r_1) = k$ & Contradiction \\
\end{tabular}

\item
\(
\begin{array}[t]{c c c c l}
(51, 15) & = & (51 - 3 \cdot 15, 15) & =\\
(6, 15) & = & (6, 15 - 2 \cdot 6) & = \\
(6, 3) & = & (6 - 2 \cdot 3, 3) & =\\
(0, 3) &  & & = & 3\\
\end{array}
\)

\item The Euclidean Algorithm:
\begin{enumerate}
\item Let $a$ and $b$ be arguments of GCD where (WLOG) $a > b > 0$.
\item Find $q_0$ and $r_0$ such that $a = b \cdot q_0 + r_0$
\item Observe $(a, b) = (b, r_1)$ by 1.33
\item Find $q_1$ and $r_1$ such that $b = r_0 \cdot q_1 + r_1$
\item Observe $(b, r_1) = (r_1, r_2)$ by 1.33
\item Starting wtih $i = 2$, until $r_i = 0$:
\begin{enumerate}
\item Find $q_i$ and $r_i$ such that $r_{i - 2} = r_{i - 1} \cdot q_{i} + r_i$
\item Observe $(r_{i - 1}, r_i) = (r_i, r_{i + 1})$ by 1.33
\item Let $i := i + 1$
\end{enumerate}
\item $r_i = 0$, therefore $(a, b) = (r{i - 1}, 0) = r_{i - 1}$
\end{enumerate}

\item 
\begin{enumerate}
\item 16
\item 1
\item 256
\item 2
\item 1
\end{enumerate}

\item $x = 9$, $y = -47$

\item The Linear Diophantine Algorithm:
\begin{enumerate}
\item Complete the EA
\item Recall the result: $r_i = 0$ and $r_{i - 1} = 1$
\item Recall the second-to-last step: $r_{i - 3} = r_{i - 2} \cdot q_{i -} + r_i$
\item Let Equation A represent: $r_{j - 2} - r_{j - 1} \cdot q_j = 1$
\item Starting with $i := i - 1$, until $i = 0$:
\begin{enumerate}
\item Justification: $r_{i - 2} = r_{i - 1} \cdot q_{i} + r_{i}$ \\
$r_{i - 2} - r_{i - 1} \cdot q_{i} = r_{i}$ \\
$r_{i}$ is a linear combination of $r_{i - 1}$ and $r_{i - 2}$
\item Substitute $r_{i}$ for $r_{i - 2} - r_{i - 1} \cdot q_{i}$ in Equation A
\item $i := i - 1$
\end{enumerate}
\item Observe that the left hand side is a linear combination of $r_0$ and $r_1$
\item Observere that the right hand side of Equation A is $1$
\item Substitute $r_1 = b - r_0 \cdot q_0$, and substitue $r_0 = a - b \cdot q_0$
\item Now a linear combination of $a$ and $b$ sums to $1$
\end{enumerate}

\item
\begin{proof}
$(a, b) = c$ & Let \\
$c|a \wedge c|b$ & Definition of GCD \\
$a = dc \fs d \wedge b = ec \fs e$ & Definition of divides \\
$ax + by = 1$ & Premise \\
$dcx + ecy = (dx + ey)c = 1$ & Algebra \\
$c = 1$ & Multiplication over integers
\end{proof}

\item 
\begin{proof}
$(a, b) = c$ & Let \\
$c|a \wedge c|b$ & Definition of GCD \\
$a = dc \fs d \wedge b = ec \fs e \wedge (d, e) = 1$ & Definition of divides \\
$\exists x, y \ni (dx + ey = 1)$ & Theorem 1.38 \\
$ax + by = dcx + ecy = (dx + ey)c = 1c =  c$ & Algebra \\
$ax + by = (a, b)$ & Substitution
\end{proof}

\item 
\begin{proof}
$bc = ka \fs k$ & Definition of divides \\
$ax + by = 1$ & 1.38 \\
$axc + byc = c = axc + kay  = c = a(xc + ky) = c$ & Algebra \\
$a|c$ & Definition of divides
\end{proof}
% \begin{proof}
% $a = a_0 a_1 \ldots a_{n} \wedge b = b_0 b_1 \ldots b_{n'} \wedge c = c_0 c_1 \ldots c_{n''}$ & Fundamental Theorem of 'rithmetic \\
% $A = \{a_0, a_1, \ldots a_{n} \} \wedge B = \{b_0, b_1, \ldots b_{n'} \} \wedge C = \{c_0, c_1, \ldots c_{n''} \}$ & Let \\
% $A \cap B = \emptyset $ & Coprime common factors lemma \\
% $A \subset (B \cup C)$ & Divisibility-subset lemma \\
% $\forall a_i \in A \{a \in (B \cup C)\}$ & Definition of subset \\
% $\forall a_i \in A \{a \in B \vee a \in C\}$ & Definition of union \\
% $\forall a_i \in A \{a \notin B\}$ & Null intersection \\
% $\forall a_i \in A \{a \in C\}$ & Disjunctive syllogism \\
% $A \subset C $ & Definition of subset \\
% $a|c$ & Subset divides superset lemma
% \end{proof}

\item 
\begin{proof}
$n = ia \fs i \wedge n = jb \fs j$ & Definition of divides \\
$ax + by = 1$ & 1.38 \\
$axn + byn = n = axjb + byia = n = ab(xj + ui) = n$ & Algebra \\
$ab|n$ & Definitin of divides
\end{proof}
% \begin{proof}
% $A = \{\textrm{factors of a}\}, B = \{\textrm{factors of b}\}, N = \{\textrm{factors of n}\}$ & Fundamental theorem of 'rithmetic \\
% $A \subset N \wedge B \subset N$ & Divisibility-subset lemma \\
% $\forall a \in A \{a \in N\} \wedge \forall b \in B \{a \in N\}$ & Definition of Subset \\
% $A \cap B = \emptyset$ & Coprime common factors lemma \\
% $\forall a \in A \{a \notin B\} \wedge \forall b \in B \{b \notin A\}$ & Null intersection \\
% $\forall a \in A \{a \notin B \wedge a \in N\} \wedge \forall b \in B \{b \notin A \wedge b \in N\}$ & Conjunction Introduction \\
% $\forall c \in (A \cup B) \{c \in N\}$ & Definition of union (without duplicates) \\
% $(A \cup B) \subset N$ & Definition of subset \\
% $ab|n$ & Divisibility-subset lemma
% \end{proof}

\item 
\begin{proof}
$ax + ny = 1 \fs x, y $\\$ bw + nz = 1\fs w, z$ & Theorem 1.38 \\
$(ax + ny)(bw + nz) = 1$ & Algebra \\
$ abxw + n(axz + ybw + yzn) = 1$ & Algebra \\
$(ab, n) = 1$ & Theorem 1.38 (converse)
\end{proof}
% \begin{proof}
% $A = \{\textrm{factors of a}\}, B = \{\textrm{factors of b}\}, N = \{\textrm{factors of n}\}$ & Fundamental theorem of 'rithmetic \\
% $A \cap N = \emptyset \wedge B \cap N = \emptyset$ & Coprime common factors lemma \\
% $\forall a \in A \{a \notin N\} \wedge \forall b \in B \{b \notin N\}$ & Null intersection \\
% $\neg(\exists a \in A \{a \in N\} \vee \exists b \in B {b \in N})$ & DeMorgan's \\
% $\neg \exists c \in (A \cup B) \{c \in N\}$ & Definition of union \\
% $\forall c \in (A \cup B) \{c \notin N\}$ & Quantificaional Negation \\
% $(A \cup B) \not\subset N$ & Definition of subset \\
% $(ab, n) = 1$ & Coprime common factors lemma
% \end{proof}

\item 
\begin{proof}
$(n, c) = 1$ & Missing hypothesis \\
$n|(ac - bc) = n|c(a - b)$ & Definition of mod \\
$n|(a - b)$ & 1.41 \\
$a \equiv b \pmod n$ & Definition of mod
\end{proof}

\item See 1.44

\item $c = k (a, b) \fs k$

\item Given integers $a$, $b$, and $c$, there exist integers $x$ and $y$ that satisfy the equation if and only if $c = k (a, b) \fs k$

\item 
\begin{proof}
Show: $ax + by = c \rightarrow (a, b)|c$ \\
$(a, b)|a \wedge (a, b)|b$ & Definition of GCD \\
$(a, b)|ax \wedge (a, b)|by$ & Theorem 1.3 \\
$(a, b)|(ax + by)$ & Theorem 1.1 \\
$(a, b)|c$ \\
Show: $(a, b)|c \leftarrow \exists x, y \{ax + by = c\}$ \\
$au + bv = (a, b)$ & Theorem 1.40 \\
$c = k(a, b)$ & Definition of divides \\
$kau + kbv = k(a, b) = c$ & Algebra \\
Putting the two halves together \\
$ax + by = c \leftrightarrow (a, b)|c$ & 
\end{proof}

\item 
The linear diophantine equation can be represented as a line on a grid. \\
$ax + by = c$ \\
$y = -\frac{a}{b}x + \frac{c}{b}$ \\
The slope of this line is $-a/b$. \\
First we must simplify the fraction: $-\frac{a}{b} = -\frac{{a} / {(a, b)}}{{b} / {(a, b)}}$ \\
 Given one point, moving $\frac{b}{(a, b)}$ on the x-coordinate to the right moves $\frac{a}{(a, b)}$ down on the y-coordinate by the properties of slope. \\
$(y - \frac{a}{(a, b)}) = -{\frac{a}{(a, b)}} / {\frac{b}{(a, b)}}(x + \frac{b}{(a, b)}) + \frac{c}{b}$ \\
$\frac{6}{(6, 15)} = 2 \wedge \frac{15}{(6, 15)} = 5$ \\
$6 \cdot (-3 + 5) + 15 \cdot (5 - 2) = 12 = 6 \cdot 2 = 12$ \\
$\forall c, d \in \mathbb{Z} \{6 \cdot (-3 + 5c) + 15 \cdot (5 - 2d) = 12\}$ \\

\item 
$\forall a, b \{31 \cdot (30 - 21a) + 21 \cdot (40 + 31b) = 1770\}$ \\

\item 
\begin{proof}
$ax_0 + by_0 = c$ & Premise \\
$a(x_0 + \frac{b}{(a, b)}) + b(y_0 - \frac{a}{(a, b)}) = ax_0 + \frac{ab}{(a, b)} + by_0 - \frac{ab}{(a, b)}$ & Distributive property \\
$ax_0 + \frac{ab}{(a, b)} + by_0 - \frac{ab}{(a, b)} = ax_0 + by_0$ & Commutative property \\
$a(x_0 + \frac{b}{(a, b)}) + y(y_0 - \frac{a}{(a, b)}) = c$ & Substitution 
\end{proof}

\item See 1.51 and 1.53

\item 
\begin{proof}
$ax + by = c$ \\
$(a, b)|a \wedge (a, b)|b$ & Definition of GCD \\
$(a, b)|c$ & Theorem 1.40 \\
$p(a, b) = c \wedge m(a, b) = a \wedge n(a, b) = b$ & Definition of divides \\
$m = \frac{a}{(a, b)} \wedge n = \frac{b}{(a, b)}$ & Algebra* \\
$(m, n) = 1$ & Lemma \\
$mx + ny = p$ & Algebra \\
$m(x + h) + n(y - k) = p \fs h, k \in \mathbb{Z}$ & Let \\
$mx + mh + ny - nk = mx + ny$ & Distributive \\
$mh = nk$ & Algebra \\
$m|mh \wedge m|nk$ & Definition of divides \\
$m|k$ & Theorem 1.41 (recall $(m, n) = 1$) \\
$k = mj \fs j \in \mathbb{Z}$ & Definition of divides* \\
$mh = nmj$ & Substitution \\
$h = nj$ & Algebra* \\
$k = \frac{aj}{(a, b)} \wedge h = \frac{jb}{(a, b)}$ & Substitution (steps with asterisks in them)
\end{proof}

\item $(24, 9) = 3$ \\
$24 \cdot 1 + 9 \cdot 1 = 33$ \\
$\forall x, y \in \mathbb{Z} \{24 \cdot (1 + 3n) + 9 \cdot (1 - 8m) = 33\}$

\item 
First without Diophantine equations:

\begin{proof}
{Show that $k \cdot \gcd(a, b)$ is a common divisor} \\
$ \gcd(a, b)|a \wedge \gcd(a, b)|b$ & Definition of GCD \\
$m \cdot \gcd(a, b) = a \fs m$ \\ $n \cdot \gcd(a, b) = b \fs n$ & Definition of divides \\
$km \cdot \gcd(a, b) = ka \wedge kn \cdot \gcd(a, b) = b$ & Algebra \\
$k \cdot \gcd(a, b)|a \wedge k \cdot \gcd(a, b)|b$ & Definition of divides \\
\hline
{Show that $k \cdot \gcd(a, b)$ is the \textbf{greatest}} \\
{common divisor by contradiction} \\
$h > k \cdot \gcd(a, b) \wedge h|ka \wedge h|kb$ & Assume (for contradiction) \\
$h = k \cdot \gcd(a, b) \cdot j \fs j$ & \textbf{Unjustified Step} \\
$(k \cdot \gcd(a, b) \cdot j)|ka \wedge (k \cdot \gcd(a, b) \cdot j)|kb$ & Substitution \\
$mjk \cdot \gcd(a, b) = ka \fs m $ \\ $njk \cdot \gcd(a, b) = kb \fs n$ & Definition of divides \\
$mj \cdot \gcd(a, b) = a \wedge nj \cdot \gcd(a, b) = b$ & Algebra \\
$j \cdot \gcd(a, b)|a \wedge j \cdot \gcd(a, b)|b$ & Definition of divides (contradicts GCD) \\
$\neg \exists h \{h > k \cdot \gcd(a, b) \wedge h|ka \wedge h|kb\}$ & Contradiction
\end{proof}

The book doesn't give a very good definition of GCD. Let $\gcd(a, b) = c$ if and only if $a = m c \fs m \in \mathbb{Z}$ and, $b = n c \fs n \in \mathbb{Z}$, and (crucially) $\gcd(m, n) = 1$

\begin{proof}
$\gcd(a, b) = c$ & Let \\
$a = cj \wedge b = ci \fs j, i \in \mathbb{Z}$ & Revised definition of GCD \\
$\gcd(i, j) = 1$ & Revised definition of GCD \\
$ka = kcj \wedge kb = kci$ & Substitution \\
$\gcd(ka, kb) = kc$ & Reivsed definition of GCD \\
& (referencing previous two steps) \\
$\gcd(ka, kb) = kc = k \cdot \gcd(a, b)$ & Substitution
\end{proof}

\item Here is my definifion of LCM. Let $a = \gcd(a, b) \cdot h \fs h \in \mathbb{Z}$ and $b = \gcd(a, b) \cdot k \fs k \in \mathbb{Z}$. I define the LCM such that $\lcm(a, b) = hk \cdot \gcd(a, b)$

\item 
\begin{proof}
$a = h \cdot \gcd(a, b) \fs h \in \mathbb{Z}$ \\
$b = k \cdot \gcd(a, b) \fs k \in \mathbb{Z}$ & Let \\
$\lcm(a, b) = h k \cdot \gcd(a, b)$ & Definition of LCM \\
$\gcd(a, b) \cdot \lcm(a, b) = h k \cdot \gcd(a, b) \cdot \gcd(a, b) = a b$ & Substitution
\end{proof}

\item
\begin{proof}
$\lcm(a, b) = ab$ & Premise \\
$\lcm(a, b) = ab \cdot \gcd(a, b)$ & Previous theorem \\
$ab \cdot \gcd(a, b) = ab$ & Substitution \\
$\gcd(a, b) = 1$ & Identity property
\end{proof}

%%% Local Variables: 
%%% mode: latex 
%%% TeX-master: "main" 
%%% End:
