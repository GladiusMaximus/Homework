\item 
\begin{proof}
$ma = b$ & Definition of `divides' \\
$na = c$ & Definition of `divides' \\
$na + ma = b + c$ & Algebra \\
$(n + m) a = b + c$ & Algebra \\
$a | (b + c)$ & Definition of `divides'
\end{proof}

\item 
\begin{proof}
Let $d = -c$ \\
$a | (b + d)$ & Theorem 1.1 \\
$a | (b - c)$ & substitution
\end{proof}

\item 
\begin{proof}
$ma = b$ & Definition of `divides' \\
$na = c$ & Definition of `divides' \\
$mana = b c$ & Algebra \\
$a | bc$ & Definition of `divides'
\end{proof}

\item 
\begin{proof}
$mana = b c$ & see last proof \\
$a^2 | bc$ & Definition of `divides'
\end{proof}

\item 
\textit{If $a|b$ then $a|b^n$}

\begin{proof}
$b = ka$ & Definition of `divides' \\
$b^n = (ka)^n = k k^{(n-1)} a^n$ & Algebra \\
$k | b^n$ & Definition of `divides'
\end{proof}

\item 
\begin{proof}
$ka = b$ & Definition of `divides' \\
$ack = bc$ & Algebra \\
$a|bc$ & Definition of `divides'
\end{proof}

\item 
\begin{enumerate}
\item $45 - 9 = 36 = 9 \cdot 4$. True
\item $37-2=35=7\cdot5$. True
\item $37-3=34$. False
\item $37 - (-3) = 40 = 8\cdot5$. True
\end{enumerate}

\item 
\begin{proof}
let $k$ be all the numbers \\
where $k \equiv b \pmod 3$ \\
$3 | (k - b)$ & Definition of `mod' \\
$3n = k - b$ & Definition of `divides' \\
$3n + k = n$ & Algebra
\end{proof}
\begin{enumerate}
\item $3n$
\item $3n + 1$
\item $3n + 2$
\item $3n$
\item $3n + 1$
\end{enumerate}

\item 
\begin{proof}
$a - a = 0 = 0n$ & Arithmetic \\
$n | (a - a)$ & Definition of `divides' \\
$a \equiv 0 \pmod n$ & Definition of `mod'
\end{proof}

\item 
\begin{proof}
$n | (a - b)$ & Definition of `mod' \\
$kn = a - b$ & Definition of `divides' \\
$-kn = b - a$ & Algebra \\
$n|(b - a)$ & Definition of `divides' \\
$b \equiv a \pmod n$
\end{proof}

\item 
\begin{proof}
$n | (a - b)$ & Definition of `mod' \\
$n | (b - c)$ & Definition of `mod' \\
$n | (a - b + b - c)$ & Theorem 1.1 \\
$n | (a - c)$ & Algebra \\
$a \equiv c \pmod n$ & Definition of `mod'
\end{proof}

\item 
\begin{proof}
$n|(a - b)$ & Definition of `mod' \\
$n|(c - d)$ & Definition of `mod' \\
$n|(a + c - b - d))$ & Theorem 1.1 \\
$n|((a + c) - (b + d))$ & Algebra \\
$a + c \equiv b + d \pmod n$ & definion `mod'
\end{proof}

\item 
\begin{proof}
let $e = -c$ and $f = -d$ \\
$a + e \equiv b + f$ & Theorem 1.12 \\
$a - c \equiv b - d$ & substitution
\end{proof}

\item 
\begin{proof}
$n | (a - b)$ & Definition of `mod' \\
$n | (c - d)$ & Definition of `mod' \\
$n | (a - b)(c - d)$ & Theorem 1.3
\end{proof}

\item 
\begin{proof}
$a \equiv b \pmod n$ & Premise \\
$a^2 \equiv b^2 \pmod n$ & Theorem 1.14
\end{proof}

\item 
\begin{proof}
$a \equiv b \pmod n$ & Premise \\
$a^2 \equiv b^2 \pmod n$ & Theorem 1.15 \\
$a^2 a \equiv b^2 b \pmod n$ & Theorem 1.14 \\
$a^3 \equiv b^3 \pmod n$ & Algebra
\end{proof}

\item 
\begin{proof}
$a \equiv b \pmod n$ & Premise \\
$a^{k - 1} \equiv b^{k - 1} \pmod n$ & Premise \\
$a^{k - 1} a \equiv b^{k - 1} b \pmod n$ & Theorem 1.14 \\
$a^k \equiv b^k \pmod n$ & Algebra
\end{proof}

\item 
\begin{proof}
Base case: \\
$a \equiv b \pmod n$ & Premise \\
Inductive Hypothesis: \\
$a^{k - 1} \equiv b^{k - 1} \pmod n$ & (assumption) \\
Inductive step: \\
$a^{k - 1} a \equiv b^{k - 1} b \pmod n$ & Theorem 1.14 \\
$a^k \equiv b^k \pmod n$ & Algebra \\
Conclusion: \\
$a^k \equiv b^k \pmod n$ & inductively
\end{proof}

\item 
\begin{enumerate}
\setcounter{enumiii}{11}
\item
$6 \equiv 2 \pmod 4$ \\
$5 \equiv 1 \pmod 4$ \\
$6 + 5 \equiv 2 + 1 \pmod 4$ \\
\item 
$6 - 5 \equiv 2 - 1 \pmod 4$ \\
\item 
$6 \cdot 5 \equiv 2 \cdot 1$ \\
\item 
$6^2 \equiv 2^2 \pmod 4$ \\
\item 
$6^3 \equiv 2^3 \pmod 4$ \\
\item 
$6^4 \equiv 2^4 \pmod 4$ \\
\item 
$6^k \equiv 2^k \pmod 4$ \\
\end{enumerate}

\item No \\
Consider the case wehre $n = 4$, $c = 0$, $a = 1$, and $b = 2$. \\
$ac \equiv bc \pmod n$ \\
$a \neq b$ \\

\item See 1.22 and 1.23

\item 
\begin{proof}
$3|a$ & Premise (Base Case) \\
$3|b$ & Let $b$ be an integer where\ldots (Inductive Hypothesis) \\
$3|9$ & Arithmetic \\
$3|(9  b_k  10^{k - 1})$ & Theorem 1.3 \\
$3|(b - 9  b_k  10^{k - 1})$ & Theorem 1.2 \\
$3|(b_{k - 1} + b_{k})b_{k - 2} \ldots b_0$ & Algebra* (Inductive Step)\\
$3|(a_{k} + a_{k - 1} + a_{k - 2} + \ldots a_1 + a_0)$ & Inductive axiom
\end{proof}

Here is the algebra I used in the step labeled `Algebra*':

\(
\begin{array}{rc}
b - b_k 9 10^{k - 1} & = \\
b - b_k (10 - 1) 10^{k - 1} & = \\
b + (-b_k  10 \cdot 10^{k - 1} + b_k  1  10^{k - 1}) & = \\
b + (-b_k  10^k + b_k  10^{k - 1}) & = \\
\begin{array}{r ccccc l}
& b_k & b_{k - 1} & b_{k - 2} & \ldots & b_0 \\
+ \quad & (-b_k) & b_k & 0 &\ldots & 0 & \quad = \\
\hline
& & (b_k + b_{k - 1}) & b_{k - 2} & \ldots & b_0 \\
\end{array}
\end{array}
\)

\item
\begin{proof}
$3|a$ & Premise (Base Case) \\
$3|(b_k + b_{k - 1} + \ldots + b_0)$ & Assumption (Inductive Hypothesis) \\
$3|9$ & Arithmetic \\
$3|(b_k 9  c)$ where c is $k$ ones in a row & Theorem 1.3 \\
$3|(b_k + b_{k - 1} + \ldots + b_0 + b_k  9  c)$ & Theorem 1.2 \\
$3|(b_k  10^k + b_{k - 1} + \ldots + b_0)$ & Algebra* \\
$3|(a_k 10^k + a_{k - 1} 10^{k - 1} + \ldots + a_0 10^0)$ & Inductive Axiom \\
$3|(a_k a_{k - 1} \ldots a_0)$ & Definition of digits
\end{proof}

Here is the algebra I used in the step labeled `Algebra*':

\(
\begin{array}{rcl}
b_k + b_{k - 1} + \ldots + b_0 + b_k  9  c & = \\
b_k + b_{k - 1} + \ldots + b_0 + b_k d & = & \textrm{where d is a number with $k$ nines} \\
b_k + b_{k - 1} + \ldots + b_0 + b_k (10^k - 1) & = \\
b_k + b_{k - 1} + \ldots + b_0 + b_k 10^k - b_k & = \\
b_{k - 1} + \ldots + b_0 + b_k 10^k
\end{array}
\)

\item
$4|a$ if and only if $4|(a_1 + a_3 + \ldots)(a_0 + a_2 + a_4 + \ldots)$

\item
\begin{enumerate}
\item $m = nq + r$ where $m = 25$, $n = 7$, $q = 3$, and $r = 4$
\item $m = 277$, $n = 4$, $q = 66$, and $r = 1$
\item $m = 33$, $n = 11$, $q = 3$, $r = 0$
\item $m = 33$, $n = 45$, $q = 0$, $r = 33$
\end{enumerate}

\item
\begin{tabular}[t]{p{4 in} l}
Setup: \\
\textit{Make a list of multiples of $n$ that are greater than $m$ and choose the smallest one to define $n(q + 1)$.} \\
$A := \{k | k \in \mathbb{N}\ \wedge kn > m \}$ &  \\
$\exists a \ni (a \in A \wedge an > m \wedge \forall k \in A (a \leq k))$ & Well-ordering Principle \\
$q : = a - 1 $ &  \\
$r := m - nq$ &  \\
\\
Proving $r$ satisfies upper bound: \\
\textit{If it didn't, then $a$ wouldn't be an element of $A$, but we know that $a$ is in $A$.} \\
$r > n - 1$ & Assume for contradiction \\
$r \geq n$ & Property of inequalities (over $\mathbb{Z}$) \\
$\exists j \ni (r - n = j \wedge j \geq 0)$ & Property of inequalities \\
$nq + r = m $ & Algebra (from definition of $r$) \\
$nq + (n + j) = m$ & Algebra (from definition of $j$)\\
$n(q + 1) + j = m$ & Algebra \\
$n(q + 1) \leq m$ & Property of inequalities \\
$n(q + 1) > m$ & Algebra (from definition of $a$) \\
$\therefore r \leq n - 1$ & Contradiction \\
\\
Proving $r$ satisfies lower bound: \\
\textit{If it didn't, then there would be another element smaller than $a$ in $A$, but $a$ is the least element in $A$.} \\
$r < 0$ & Assume for contradiction \\
$nq + r = m $ & Algebra (from definition of $r$) \\
$nq > m$ & Property of inequalities \\
$q \in A$ & $q \in \mathbb{N} \wedge nq > m$ is the condition for $A$ \\
$\forall k (k \in A \rightarrow q + 1 \leq k)$ & Definition of $a$ (smallest element in $A$)\\
$q + 1 \leq q$ & Universal instantiation \\
$\therefore r \geq 0$ & Contradiction \\
\\
Proving $q$ and $r$ are integers: \\
\textit{They all came from sets that only contain integers.} \\
$A \subset \mathbb{N} \subset \mathbb{Z}$ & Stuff I learned \\
$a \in A$ & Definition of $a$ \\
$a \in \mathbb{Z}$ & Property of sets \\
$q \in \mathbb{Z}$ & Closure (Definition of $q$) \\
$r \in \mathbb{Z}$ & Closure (definition of $r$) \\
\\
\end{tabular}

\item 
\begin{tabular}[t]{p{3.5in} p{3in}}
$\exists q', r' \in \mathbb{Z} (m = q'n + r' \wedge r' \neq r \wedge q' \neq q \wedge 0 \leq r \leq q' - 1)$ & Assume for contradiction \\
$r' < n$ & Assumption (restriction on $r'$) \\
$q'n + n > m$ & Property of inequalities (because $q'n + r = m$)\\
$n(q' + 1) > m$ & Algebra \\
$q' + 1 \in A$ & Definition of $A$ \\
$q' + 1 \neq q + 1$ & Property of inequalities \\
$q' + 1 > q + 1$ & Definition of $a$ (smallest element in $A$) \\
$q' \geq q + 1$ & Property of inequalities (over $\mathbb{Z}$) \\
$qn + r = m$ & Definition of $r$ \\
$qn + n > m$ & Property of inequalities (replace $r$ with something greater-than $r$) \\
$(q + 1)n > m$ & Algebra \\
$q'n > m$ & Property of inequalities (replace $q + 1$ with something greater-than-or-equal to it) \\
$q'n + r' > m$ & Property of inequalities (add a positive number to the bigger side and it is still bigger) \\
$\neg \exists q', r' \in \mathbb{Z} (m = q'n + r' \wedge r' \neq r \wedge q' \neq q \wedge 0 \leq r \leq q' - 1)$ & Contradiction \\
\end{tabular}

\item  Yes. $1$

\item No. There are a finite number of integer factors.

\item
\begin{enumerate}
\item No
\item No
\item No
\item Yes
\item Yes
\item Yes
\end{enumerate}

\item
\begin{proof}
$a - nb = r$ & Algebra (from premise)\\
$k|nb$ & Theorem 1.3 \\
$k|(a - nb)$ & Theorem 1.2 \\
$k|r$ & Subsitution
\end{proof}

\item 
\begin{proof}
$(a, b) = k$ & Let \\
$k|a$ & Definition of $k$ (GCD) \\
$k|b$ & Definition of $k$ (GCD) \\
$k|r_1$ & Theorem 1.32 \\
\hline
$n_1 = 0$ & Assume temporarily
$a = r_1$
$(a, b) = (b, r_1)$ & Substitution \\
\hline
$n_1b > 0$ & Assume temporarily
$a > r_1$

\hline

\end{proof}

%%% Local Variables: 
%%% mode: latex 
%%% TeX-master: "main" 
%%% End: