\setcounter{enumii}{10}
\item \textbf{Theorem:} Let \(f\) be an \(n\)-degree monic polynomial such that \(f(x) = a_n x^n + a_{n-1} x^{n-1} + \dots + a_0 \). \(\exists k \in \mathbb N (\forall x > k (f(x) > 0))\).

\textbf{Proof:} \(x > \lvert a_{n-1} \rvert\) is sufficient for \(x^n > a_{n-1} x^{n-1}\). That is because multiplying both sides of the condition by \(x^{n-1}\) (valid operation since \(x^{n-1}>0\), since \(x>0\)) gives \(x x^{n-1} > a_{n-1} x^{n-1}\), equivalently \(x^n  > a_{n-1} x^{n-1}\). That simply arises from the initial condition. After this point, the \(n\)th term dominates the \((n-1)\)th term.

If the first term dominates the zeroth term at some point \(k_1\), and the second term dominates the first term at some point \(k_2\), then at some point greater than \(k_1\) and greater than \(k_2\), the third term dominates the second term and the second term dominates the first term (\(\lvert a_2 x^2 \rvert > \lvert a_1 x \rvert > \lvert a_0 \rvert\)). Therefore the third term dominates the first term (\(\lvert a_2 x^2 > a_0\)). 

Continuing in this way, there is some point \(k_n\) the \(n\)th term dominates the \((n-1)\)th term. The \((n-1)\)th term dominates the \((n-2)\)th term after \(k_{n-1}\). Therefore for \(x > k\) where \(k = \max(k_n, k_{n-1}, \dots, k_1)\), the \(n\)th term dominates. Since the polynomial is monic, \(a_n > 0\). Therefore \(\lvert a_n x^n \rvert > \lvert a_{n-1} x^{n-1} \rvert > \dots > \rvert a_0 \lvert\). Therefore \(n \lvert a_n x^n \rvert > \lvert a_{n-1} x^{n-1} \rvert + \dots + \rvert a_0 \lvert\). 

\setcounter{enumii}{13}
\item \textbf{Theorem:} \(\forall i \in \mathbb Z (\forall j \in \mathbb N (\exists ! r \in \mathbb N (i \equiv r \pmod j \wedge 0 \leq r < j)))\)

\begin{proof}
Let \(i \in \mathbb N\) & (for universal generalization) \\
Let \(j \in \mathbb N\) & (for universal generalization) \\
If \(i > 0\) \\
Conclude: \(\exists ! q, r \in \mathbb N (i = qj + r \wedge 0 \leq r < j) \) & Division algorithm \\
Otherwise \(i < 0\) \\
\(\exists ! p, r \in \mathbb N (-i = pj + t \wedge 0 \leq t < j) \) & Division algorithm \\
\(-i = pj + t \wedge 0 \leq t < j \) & Existential generalization \\
\(i = -pj - t\) & Existential generalization \\
\(i = -pj - j + j - t\) & Algebra \\
\(i = -(p+1)j + j - t\) & Algebra \\
\(0 \leq t < j\) & Simplification \\
\(-j < -t \leq 0\) & Property of inequalities \\
\(0 < j - t \leq j\) & Property of inequalities \\
If \(j - t < j\) \\
Let \(q = -(p+1)\)
Let \(r = j - t\) \\
\(0 < r < j\) & Property of inequalities \\
\(0 \leq r < j\) & Property of inequalities \\
Conclude: \(\exists ! q, r \in \mathbb N (i = qj + r \wedge 0 \leq r < j) \) & Existential generalization \\
Otherwise \(j - t \geq j\) \\
\(j - t \leq j \wedge j -t \geq j\) & Conjunction \\
\(j - t = j\) & Property of inequalities \\
\(t = 0\) & Identity property \\
\(i = pj\)
Let \(r = 0\) \\
Conclude: \(\exists ! q, r \in \mathbb N (i = qj + r \wedge 0 \leq r < j) \) & Existential generalization \\
\(\exists ! q, r \in \mathbb N (i = qj + r \wedge 0 \leq r < j) \) & Constructive dilemma \\
Conclude: \(\exists ! q, r \in \mathbb N (i = qj + r \wedge 0 \leq r < j) \) & Constructive dilemma \\
\(\forall i \in \mathbb N (\forall j \in \mathbb N (\exists ! r \in \mathbb N (i \equiv r \pmod j \wedge 0 \leq r < j)))\) & Universal generalization \\ & (used twice)
\end{proof}

\item
\begin{enumerate}
\item \(\{0, 1, 2, 3\}\)
\item \(\{-4, -3, -2, -1\}\)
\item \(\{0, 5, 10, 15\}\)
\end{enumerate}

Let \(A \in \crs(n)\) stand for \(A\) is a possible Complete Residue System (CRS) for mod \(n\).

Let \(A \in \ccrs(n)\) stand for \(A\) is the Canonical Complete Residue System (CCRS) for mod \(n\).

\item \textbf{Theorem:} \(B \in \crs (n) \rightarrow \lvert B \rvert = n\)

\begin{proof}
Let \(A \in \ccrs(n)\) \\
Let \(B \in \crs(n)\) & For conditional\\
Let \(f : A \to B\) where \(a \mapsto b\) if \(a \equiv b \pmod n\) \\
\(\forall a \in A (\exists! b \in B (x \equiv b \pmod n))\) & Definition of CRS \\
\(\forall a \in \cod (f) (\exists ! b \in \dom (f) (f(a) = b))\) & Substitution \\
Thus \(f\) is a bijective map \\
\(\lvert A \rvert = n\) & By inspection \\
Thus \(\lvert A \rvert = \lvert B \rvert = n\) & Bijection \\
\(B \in \crs (n) \rightarrow \lvert B \rvert = n\) & Conditional proof
\end{proof}

\item \textbf{Theorem:} \(\neg \exists a \in S (\exists b \in S (a \equiv b \pmod n \wedge a \neq b)) \rightarrow S \in \crs(n)\)

Let \(\rem(x \pmod n)\) (read ``remainder of x modulo n'')denote the number in the Complete Canonical Residue System congruent to \(x\) mod \(n\).

\textbf{Lemma: } \(a = b \rightarrow a \equiv b \pmod n\)
\begin{proof}
\(a - b = 0\) & Algebra \\
\(0n = 0\) & Zero-property of multiplication \\
\(n \mid (a - b)\) & Definition of divides \\
\(a \equiv b \pmod n\) & Definition of modulo
\end{proof}


\begin{proof}
%\(\forall i \in \mathbb S (\exists ! r \in \mathbb N (i \equiv r \pmod n \wedge 0 \leq r < n))\) & Theorem 3.14 \\
Assume \(\neg \exists a \in S (\exists b \in S (a \equiv b \pmod n \wedge a \neq b))\) & (for conditional) \\
Assume \(\exists a \in S (\exists b \in S (\rem(a \pmod n) = \rem(b \pmod n)))\) & (for contradiction) \\
\(a \equiv \rem(a \pmod n)\) & Definition of remainder \\
\(b \equiv \rem(b \pmod n)\) & Definition of remainder \\
\(a \equiv \rem(a \pmod n) \equiv b\) & Lemma and transitivity \\
\(\exists a \in S (\exists b \in S (a \equiv b \pmod n \wedge a \neq b))\) & Existential generalization \\
\(\neg \exists a \in S (\exists b \in S (\rem(a \pmod n) = \rem(b \pmod n)))\) & Contradiction \\
\(\)
\end{proof}

\item 
\begin{enumerate}
\item \(x \equiv 1 \pmod 3\)
\item \(x \equiv 4 \pmod 5\)
\item No solution.
\item \(x \equiv 14 + 71n \pmod{213}\) for \(n \in \{0, 1, 2\}\)
\end{enumerate}

\item \textbf{Theorem:} \(\exists x \in \mathbb Z (ax \equiv b \pmod n) \leftrightarrow \exists x, y \in \mathbb Z (ax - ny = b)\)
%TODO: deal with existence

\begin{proof}
\(\exists x \in \mathbb Z (ax \equiv b \pmod n) \leftrightarrow \exists x \in \mathbb Z (b \equiv ax \pmod n)\) & Theorem 1.10 \\
\(\exists x \in \mathbb Z (b \equiv ax \pmod n) \leftrightarrow \exists x \in \mathbb Z (n\mid(b - ax))\) & Definition of modulo \\
\(\exists x \in \mathbb Z (n\mid(b - ax)) \leftrightarrow \exists x, y \in \mathbb Z (ny = b - ax)\) & Definition of divides \\
\(\exists x, y \in \mathbb Z (ny = b - ax) \leftrightarrow \exists x, y \in \mathbb Z (ax + ny = b)\) & Algebra \\
\(\exists x \in \mathbb Z (ax \equiv b \pmod n) \leftrightarrow \exists x, y \in \mathbb Z (ax - ny = b)\) & Transitivity
\end{proof}

\item \textbf{Theorem:} \(\exists x \in \mathbb Z (ax \equiv b \pmod n) \leftrightarrow \gcd(a, n)\mid b \)

\begin{proof}
\(\exists x \in \mathbb Z (ax \equiv b \pmod n) \leftrightarrow \exists x, y \in \mathbb Z (ax - ny = b)\) & Theorem 3.19 \\
\(\exists x, y \in \mathbb Z (ax - ny = b) \leftrightarrow \gcd(a, n) \mid b\) & 1.48 \\
\(\exists x \in \mathbb Z (ax \equiv b \pmod n) \leftrightarrow \gcd(a, n)\mid b \) & Transitivity
\end{proof}

\item It has a solution.

\item
\begin{tabular}[t]{l}
\(213 - 8 \cdot 24 = 21\) \\
\(24 - 1 \cdot 21 = 3\) \\
\(24 - 1 \cdot (213 - 8 \cdot 24) = 3\) \\
\(9 \cdot 24 - 213 = 3\) \\
\(41 \cdot (9 \cdot 24 - 213) = 41 \cdot 3 = 123\) \\
\(369 \cdot 24 - 41 \cdot 213 = 123\) \\
\((369 + n \cdot 71) \cdot 24 - (41 + n \cdot 8) \cdot 213 = 123\) \\
\(213 \mid ((369 + n \cdot 71) \cdot 24 - 213)\) \\
\(x = 369 + n \cdot 71\)
\end{tabular}

\item \textbf{Algorithm: } Find all solutions of \(ax = b \pmod n\) for \(0 \leq x < n\)

\textbf{Steps: } \\
\begin{enumerate}
\item WLOG \(a < n\), otherwise reduce \(a\).
\item Let \(r_1 := q_0 n - a\) with \(0 \leq r_1 < n\) by the Division algorithm.
\item Let \(r_2 := q_1 a - r_1\) with \(0 \leq r_1 < a\) by the Division algorithm.
\item Starting with \(i = 2\), repeating until \(r_{i+2} = 0\)
\begin{enumerate}
\item Let \(r_{i+1} := r_{i-1} - q_i r_i\) with \(0 \leq r_{i+1} < r_i\) by the Division algorithm.
\item Let \(i := i + 1\)
\end{enumerate}
\item \(r_{i+1} = \gcd(n, a)\) by the argument in 2.35
\item Observe that \(\gcd(n, a)= r_{i+1} = r_{i-1} - q_i r_i\) (from assignment of \(r_{i+1}\))
\item Starting with \(j = i - 1\), until \(j = 1\)
\begin{enumerate}
\item Replace \(r_{j + 1}\) with \(r{j-1} - q_j r_j\) (from the assignment of \(r_{i+1}\))
\item Let \(j := j - 1\)
\item Observe that \(r_{j}\) is a linear combination of \(r_{j-1}\) and \(r_j\)
\end{enumerate}
\item Subsitute \(r_1\) with \(q_0 n - b\) and \(r_2\) with \(q_1 a - r_1\)
\item Since \(\gcd(n, a) = r_{i+1}\), and \(r_{i+1}\) is written as a linear combination of \(r_i\) and \(r_{i-1}\), and \(r_1\) and \(r_2\) are written as a linear combinatino of \(a\) and \(b\), \(gcd(n, a)\) is written as a linear combination of \(a\) and \(b\) after substitution. Let that combination be \(ax + ny = b\)
\item Therefore \(\frac{\gcd(n, a)}{b} ax + \frac{\gcd(n, a)}{b} ny = \frac{\gcd(n, a)}{b} b = b\) by algebra with additional solutions are found at \((\frac{\gcd(n, a)}{b} x + m\frac{n}{\gcd(n, a)})a + (\frac{\gcd(n, a)}{b} y - m\frac{a}{\gcd{n, a}})n = b\) by Theorem 1.51.
\item Therefore solution is found at \(x = \frac{\gcd(n, a)}{b} a + m\frac{n}{\gcd(n, a)}\)
\tiny {$~\blacksquare$}
\end{enumerate}

\textbf{Theorem:} There are \(\frac{n}{\gcd(a, n)}\) solutions to the linear congruence.

\begin{proof}
\(0 \leq x_0 < \frac{n}{\gcd(a, n)}\) \\
\(0 + (\gcd(a, n) - 1) \frac{n}{\gcd(a, n)}
\leq x_0 + (\gcd(a, n) - 1)\frac{n}{\gcd(a, n)}
< \frac{n}{\gcd(a, n)} + (\gcd(a, n) - 1)\frac{n}{\gcd(a, n)}
\) & Addition property of inequalities \\
\(0 + (\gcd(a, n) - 1) \frac{n}{\gcd(a, n)}
\leq x_0 + (\gcd(a, n) - 1)\frac{n}{\gcd(a, n)}
< \frac{n}{\gcd(a, n)} + \gcd(a, n)\frac{n}{\gcd(a, n)} - \frac{n}{\gcd(a, n)}
\) & Distributive property \\
\((\gcd(a, n) - 1) \frac{n}{\gcd(a, n)}
\leq x_0 + (\gcd(a, n) - 1)\frac{n}{\gcd(a, n)}
< \gcd(a, n)\frac{n}{\gcd(a, n)}
\) & Identity \\
For all  \(0 \leq m \leq \gcd(a, n) - 1\), there are solutions at \(x_0 + m \frac{n}{\gcd(a, n)}\) in the CCRS \\
There are \(\gcd(a, n)\) solutions
\end{proof}

\item 3.20, 3.23a, and 3.23b taken together prove this theorem. The big idea is that a linear congruence is a special kind of linear diophantine equation.

\item Solve for \(x\) in

\(
\begin{array}[t]{l}
x \equiv 3 \pmod {17} \\
x \equiv 10 \pmod {16} \\
x \equiv 0 \pmod {15} \\
\end{array}
\)

\begin{tabular}[t]{p{7in}}
\(x = \{\)\(0\), \(1\), \(2\), \(3\), \(4\), \(5\), \(6\), \(7\), \(8\), \(9\), \(10\), \(11\), \(, \dots\}\) \\
\\ \(x\) satisfies \(1 x \equiv 3 \pmod {17}\) and all previous equations
when \(x = 3 + j \cdot 17\) \\
\(x = \{\)\(3\), \(20\), \(37\), \(54\), \(71\), \(88\), \(105\), \(122\), \(139\), \(156\), \(173\), \(190\), \(207\), \(224\), \(241\), \(258\), \(275\), \(292\), \(309\), \(326\), \(343\), \(360\), \(377\), \(394\), \(, \dots\}\) \\
\\ \(x\) satisfies \(1 x \equiv 10 \pmod {16}\) and all previous equations
when \(x = 122 + j \cdot 272\) \\
\(x = \{\)\(122\), \(394\), \(666\), \(938\), \(1210\), \(1482\), \(1754\), \(2026\), \(2298\), \(2570\), \(2842\), \(3114\), \(3386\), \(3658\), \(3930\), \(4202\), \(4474\), \(4746\), \(5018\), \(5290\), \(5562\), \(5834\), \(6106\), \(6378\), \(6650\), \(6922\), \(7194\), \(7466\), \(7738\), \(8010\), \(8282\), \(8554\), \(8826\), \(9098\), \(9370\), \(9642\), \(9914\), \(10186\), \(10458\), \(10730\), \(11002\), \(11274\), \(11546\), \(11818\), \(12090\), \(, \dots\}\) \\
\\ \(x\) satisfies \(1 x \equiv 0 \pmod {15}\) and all previous equations
when \(x = 3930 + j \cdot 4080\) \\
\end{tabular}

\item Solve for \(x\) in

\(
\begin{array}[t]{l}
x \equiv 1 \pmod 2 \\
x \equiv 2 \pmod 3 \\
x \equiv 3 \pmod 4 \\
x \equiv 4 \pmod 5 \\
x \equiv 5 \pmod 6 \\
x \equiv 0 \pmod 7 \\
\end{array}
\)

\begin{tabular}[t]{p{7in}}
\(x = \{\)\(0\), \(1\), \(2\), \(3\), \(4\), \(5\), \(, \dots\}\) \\
\\ \(x\) satisfies \(1 x \equiv 1 \pmod {2}\) and all previous equations
when \(x = 1 + j \cdot 2\) \\
\(x = \{\)\(1\), \(3\), \(5\), \(7\), \(9\), \(11\), \(13\), \(15\), \(17\), \(, \dots\}\) \\
\\ \(x\) satisfies \(1 x \equiv 2 \pmod {3}\) and all previous equations
when \(x = 5 + j \cdot 6\) \\
\(x = \{\)\(5\), \(11\), \(17\), \(23\), \(29\), \(35\), \(, \dots\}\) \\
\\ \(x\) satisfies \(1 x \equiv 3 \pmod {4}\) and all previous equations
when \(x = 11 + j \cdot 12\) \\
\(x = \{\)\(11\), \(23\), \(35\), \(47\), \(59\), \(71\), \(83\), \(95\), \(107\), \(119\), \(131\), \(143\), \(155\), \(167\), \(179\), \(, \dots\}\) \\
\\ \(x\) satisfies \(1 x \equiv 4 \pmod {5}\) and all previous equations
when \(x = 59 + j \cdot 60\) \\
\(x = \{\)\(59\), \(119\), \(179\), \(, \dots\}\) \\
\\ \(x\) satisfies \(1 x \equiv 5 \pmod {6}\) and all previous equations
when \(x = 59 + j \cdot 60\) \\
\end{tabular}

\item \textbf{Theorem:} Let \(a, b, m, n \in \mathbb Z\) where \(m > 0\) and \(n > 0\). The system \(x \equiv a \pmod n\) and \(x \equiv b \pmod m\) has solutions for \(x\) if and only if \(\gcd(n, m) \mid (a - b)\)

\textbf{Proof:} \(x \equiv a \pmod m\), or equivalently \(m \mid (x-a)\), or equivalently, \(cm = x - a\), and by the same logic \(dn = x - b\). Adding the system of equations together, \(cm - dn = x - a - (x - b)\), or equivalently \(cm - dn = a - b\). By Theorem 1.48, this has solutions if and only if \(\gcd(m, n) \mid (a - b)\).

\item \textbf{Theorem:} Let \(a, b, m, n \in \mathbb Z\) where \(m > 0\), \(n > 0\), and \(\gcd(m, n) = 1\)

\textbf{Proof:} Repeat the previous proof up to \(cm - dn = a - b\).  \(c\) has solutions every \(\frac{n}{\gcd(m, n)} = n\) and \(d\) has solutions every \(\frac{m}{\gcd(m, n)} = m\). \(a + cm = x\) and \(b + dn = x\). \(x = a + m(c_0 + in) = a + m c_0 + inm\) and \(x = b + n(d_0 + im) = b + nd_0 + inm\). Solving for \(x\) in terms of \(c\) and solving for \(x\) in terms of \(d\) both indicate solutions every \(nm\). Therefore they are equivalent to the same thing mod \(mn\)


\setcounter{enumi}{4}
\setcounter{enumii}{0}

\item 
\(
\begin{array}[t]{ll}
2^0 \pmod 7 & 1 \\
2^1 \pmod 7 & 2 \\
2^2 \pmod 7 & 4 \\
2^3 \pmod 7 & 1 \\
2^4 \pmod 7 & 2 \\
2^5 \pmod 7 & 4 \\
2^6 \pmod 7 & 1 \\

\end{array}
\)

%%% Local Variables:
%%% mode: latex
%%% TeX-master: "main"
%%% End:
