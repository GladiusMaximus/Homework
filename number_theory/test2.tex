\documentclass[12pt,letterpaper]{article}
\usepackage[utf8]{inputenc}
%\usepackage{ifpdf,mla}
%\usepackage{showframe}
\usepackage[english]{babel}
\usepackage{amsmath}
\usepackage{amssymb}
\usepackage[letterpaper, margin=1in]{geometry}
\usepackage{setspace}
\usepackage{enumitem}
\usepackage[pdftex]{graphicx}
\usepackage{longtable}
\usepackage{calc}
\usepackage{fancyhdr}
\usepackage{lastpage}
\usepackage{cancel}

\pagestyle{fancy}
\fancyhead{}
\fancyfoot{}
\fancyhead[HR]{\thepage\ of \pageref{LastPage}}

\newcommand{\fs}{\textrm{\ for some\ }}
\DeclareMathOperator{\lcm}{lcm}
\DeclareMathOperator{\pf}{pf}
\newenvironment{proof}{
\textbf{Proof:} \\
\mbox{}\vspace*{-1.68\baselineskip}
\setlength\LTleft{\leftmargin+20pt}
\setlength\LTright\fill
\begin{longtable}{@{} ll}
}{
\tiny {$~\blacksquare$}
\end{longtable}
}

\begin{document}

% Definition of Hilbert number
% 5.a: explain process

\doublespacing
\begin{center}
{\Large Test 2} \\[14pt]
{\large Sam Grayson} \\[0pt]
{\today} \\
\end{center}

\singlespacing
\setlength{\parindent}{0pt}

Before beginning the my answers, I need to establish the following lemma.

\textbf{Lemma: } \(\forall i \in \mathbb N (\forall j \in \mathbb N (\exists ! r \in \mathbb N (i \equiv r \pmod{j} \wedge 0 \leq r < j)))\)

\begin{proof}
Let \(i \in \mathbb N\) & (for universal generalization) \\
Let \(j \in \mathbb N\) & (for universal generalization) \\
\(i = qj + r \wedge 0 \leq r < j \textrm{\ for some unique\ } q, r \in \mathbb N\) & Division algorithm \\
\(i - r = qj\) & Algebra \\
\(j|(i - r)\) & Definition of divides \\
\(i \equiv r \pmod j\) & Definition of modulo \\
\(i \equiv r \pmod j \wedge 0 \leq r < j\) & Conjunction \\
\(\exists r (i \equiv r \pmod j \wedge 0 \leq r < j)\) & Existential generalization \\
\(\forall i \in \mathbb N (\forall j \in \mathbb N (\exists ! r \in \mathbb N (i \equiv r \pmod j \wedge 0 \leq r < j)))\) & Universal generalization
\end{proof}

\begin{enumerate}[leftmargin=0mm]
%\renewcommand{\theenumi}{\Alph{enumi}}
\item \textbf{Theorem:} \(\sqrt[3] 6 \notin \mathbb{Q}\)

\begin{proof}
\(\sqrt[3] 6 \neq 1\) & Fact \\
\(\sqrt[3] 6 \neq 2\) & Fact \\
\(\sqrt[3] 6 \neq 3\) & Fact \\
\(\sqrt[3] 6 \neq 4\) & Fact \\
\(\sqrt[3] 6 \neq 5\) & Fact \\
\(\sqrt[3] 6 \neq 6\) & Fact \\
\(\forall x \in \mathbb N (1 \leq \sqrt[3] x \leq x)\) & Fact \\
\(1 \leq \sqrt[3] 6 \leq 6\) & Universal instantiation \\
Ugggggh \\
\(\sqrt[3] 6 \notin \mathbb N\) & By exhaustion \\
\(\sqrt[3] 6 \notin \mathbb Q\) & Next test question
\end{proof}

\item \textbf{Theorem:} Let \(n, x \in \mathbb{N}\). \(\sqrt[n]{x} \in \mathbb{Q} \rightarrow \sqrt[n]{x} \in \mathbb{N}\).

\begin{proof}
\(\sqrt[n]{x} \in \mathbb{Q}\) & Premise \\
\(\sqrt[n]{x} = \frac{j}{k} \fs j, k \in \mathbb{Z}\) & Definition of rational \\
\(x k^n = j^n\) & Algebra \\
\(x k_0^n k_1^n k_2^n \ldots = j_0^n j_1^n j_2^n \ldots\) & FTA \\
\(x k_1^n k_2^n \ldots = j_1^n j_2^n \ldots\) & Theorem 2.8 (with reordering) \\
\(x k_2^n \ldots = j_2^n \ldots\) & Theorem 2.8 (with reordering) \\
Repeating this process \\
Stop when all \(k\) are eliminated \\
Lets call it the \(i\)th step \\
\(x = j_i^n j_{i+1}^n \ldots\) & Theorem 2.8 \\
\(\sqrt[n]{x} = j_i j_{i + 1} \ldots\) & Algebra \\
\(\sqrt[n]{x} \in \mathbb{N}\) & Closure of \(\mathbb{N}\) over multiplication
\end{proof}

\item \textbf{Theorem:}: Let \(n \equiv 2 \pmod 3\). Let the prime-factorization of \(n\) be written as follows: \(n = n_1 n_2 n_3 \dots\). I claim that \(\exists i (n_i \equiv 2)\)

\begin{proof}
\multicolumn{2}{l}{Note: all statements of congruency are taken to be modulo 3.} \\
Assume \(\forall i (n_i \not \equiv 2)\) & For contradiction \\
\(\forall i (i \equiv 0 \vee i \equiv 1 \vee i \equiv 2))\) & Lemma \\
\(\forall i (i \not \equiv 2 \rightarrow (i \equiv 2 \vee i \equiv 0))\) & Conditional disjunction \\
\(n_i \not \equiv 2\) & Universal instantiation \\
\(n_i \not \equiv 2 \rightarrow (n_i \equiv 1 \vee n_i \equiv 0)\) & Universal instantiation \\
\(n_i \equiv 1 \vee n_i \equiv 0\) & Modus ponens \\
\(\forall i (n_i \equiv 1 \vee n_1 \equiv 0)\) & Universal generalization \\
\(n_0 n_1 n_2 \ldots \equiv 1^i 0^j \fs i, j \in \mathbb N\) & Theorem 1.14 (repeated use) \\
If \(i = 0 \wedge j \neq 0\): \(n \equiv 0\) & Arithmetic \\
If \(i \neq 0 \wedge j = 0\): \(n \equiv 1\) & Arithmetic \\
If \(i \neq 0 \wedge j \neq 0\): \(n \equiv 0\) & Arithmetic \\
If \(i = 0 \wedge j = 0\): \(n \equiv 1\) & Arithmetic \\
\(n \equiv 0 \vee n \equiv 1\) & Constructive dilemma \\
Contradicts \(n \equiv 2\) & \\
\(\neg (\forall i (n_i \not \equiv 2))\) & Contradiction \\
\(\neg (\neg \exists i (n_i \equiv 2))\) & Quantifier exchange \\
\(\exists i (n_i \equiv 2)\) & Double negation
\end{proof}

\item 
\begin{enumerate}
\item Theroem: \((a \in H \wedge b \in H) \rightarrow ab \in H\).

\begin{proof}
\(a = 4c + 1\) & Definition of Hilbert number \\
\(b = 4d + 1\) & Definition of Hilbert number \\
\(ab = 16cd + 4c + 4d + 1 = 4(4cd + c + d) + 1\) & Algebra \\
\(ab \equiv 1 \pmod 4\) & Definition of modulo \\
\(ab \in H\) & Definition of \(H\)
\end{proof}


\item 5, 9, 13, 17, 21, 29, 33, 37, 41, 49

\item Show: for all \(a \in H\), \(a\) can be factored into elements of \(H\)

Some Hilbert numbers \(a\) are divisible by some other Hilbert element \(b \in H\). In order for \(a\) to factor, it has to be written as \(a = bc\) for \(c \in H\). In other words, \(\forall a \in H (\forall b \in H (b|a \rightarrow \exists c \in H (bc = a)))\). (These are the Hilbert composites)

\begin{proof}
\multicolumn{2}{l}{Note: all statements of congruency are taken to be modulo 4.} \\
Let \(a \in H\) & Assume (for universal generalization) \\
Let \(b \in H\) & Assume (for universal generalization) \\
Let \(b|a\) & Assume (for conditional) \\
\(bc = a \fs c \in \mathbb N\) & Definition of divides \\
& and existential instantiation (on \(c\)) \\
\(a = 4k_a + 1 \fs k_a\) & Definition of Hilbert number \\
\(b = 4k_b + 1 \fs k_b\) & Definition of Hilbert number \\
\(a - 1 = 4k_a\) & Algebra \\
\(b - 1 - 4k_b\) & Algebra \\
\(4|(a - 1)\) & Definition of divides \\
\(4|(b - 1)\) & Definition of divides \\
\(a \equiv 1\) & Definition of modulo \\
\(b \equiv 1\) & Definition of modulo \\
\(\forall i (i \equiv 0 \vee i \equiv 1 \vee i \equiv 2 \vee i \equiv 3)\) & Lemma \\
Assume \(c \not \equiv 1\) & For contradiction \\
\(i \equiv 0 \vee i \equiv 1 \vee i \equiv 2 \vee i \equiv 3\) & Universal instantiation \\
\(c \not \equiv 1 \rightarrow (c \equiv 0 \vee c \equiv 2 \vee c \equiv 3)\) & Universal generalization \\
\(i \not \equiv 1 \rightarrow (i \equiv 0 \vee i \equiv 2 \vee i \equiv 3)\) & Conditional disjunction \\
\(c \equiv 0 \vee c \equiv 2 \vee c \equiv 3\) & Modus ponens \\
If \(c \equiv 0\): \(bc \equiv 1 \cdot 0 \equiv 0\) & Theorem 1.14 \\
If \(c \equiv 2\): \(bc \equiv 1 \cdot 2 \equiv 2\) & Theorem 1.14 \\
If \(c \equiv 3\): \(bc \equiv 1 \cdot 3 \equiv 3\) & Theorem 1.14 \\
\(bc \equiv 0 \vee bc \equiv 2 \vee bc \equiv 3\) & Constructive dilemma \\
Contradicts \(bc \equiv 1\) \\
\(c \equiv 1\) & Contradiction \\
\(c \in H\) & Definition of Hilbert number \\
\(\exists c \in H (bc = a)\) & Existential generalization \\
\(b|a \rightarrow \exists c \in H (bc = a)\) & Conditional proof \\
\(\forall b \in H (b|a \rightarrow \exists c \in H (bc = a))\) & Universal generalization \\
\(\forall a \in H \forall b \in H (b|a \rightarrow \exists c \in H (bc = a))\) & Universal generalization
\end{proof}

Every Hilbert number that is not divisible by another Hilbert number can be written \(a = 1a\), since \(1 \in H\) and \(a \in H\). (These are the Hilbert primes).

Therefore every Hilbert number can be factored into other Hilbert numbers,

\item
\(
\begin{array}[t]{r}
693 = 9 \cdot 77 \\
693 = 21 \cdot 33 \\
\end{array}
\)

\end{enumerate}


\item 
\begin{enumerate}
\item Using finite-list notation, the GCD is equivalent to the list-intersection. Since 3 occurs in the prime-factorization of \(a\) 3 times and in the prime-factorization of \(b\) 2 times, 3 occurs in the prime factorization of the GCD \(\min(3, 2) = 2\).

\begin{tabular}[t]{l}
\(A = \pf(a) = \{2, 2, 2, 3, 3, 5, 5, 13\}\) \\
\(B = \pf(b) = \{2, 2, 3, 3, 3, 13, 13, 13, 13, 13, 13, 13, 19\}\) \\
\(\min(2 \# A, 2 \# B) = \min(3, 2) = 2\) \\
\(\min(3 \# A, 3 \# B) = \min(2, 3) = 2\) \\
\(\min(13 \# A, 13 \# B) = \min(1, 7) = 1\) \\
\(\pf(a) \cap \pf(b) = \{2, 2, 3, 3, 13\}\) \\
\(\gcd(a, b) = \prod (\pf(a) \cap \pf(b)) = 156\) \\
\end{tabular}
\item
\begin{tabular}[t]{l}
\(A = \pf(a) = \{2, 2, 2, 3, 3, 5, 5, 13\}\) \\
\(B = \pf(b) = \{2, 2, 3, 3, 3, 13, 13, 13, 13, 13, 13, 13, 19\}\) \\
\(\pf(6a^2 + 5b^3) = \{2, 3\} + 2A + \{5\} + 3B\) \\
\(2 \# \pf(6a^2 + 5b^3) = 2 \# (\{2, 3\} + 2A + \{5\} + 3B\)) \\
\(\quad = 1 + 2 \cdot 3 + 0 + 3 \cdot 2\) \\
\(\quad = 13\) \\
\(2^13 \mid (6 a^2 + 5 b^3)\) \\
\end{tabular}
\end{enumerate}

\item \textbf{Theorem:} Let \(n \in \mathbb Z \wedge n > 0\). Let the smallest prime factor be \(p\). \(p > \sqrt[3] n \rightarrow (\frac{n}{p} = 1 \vee \frac n p \in \mathbb P)\)

Consider the case where \(n\) is composite. \(\frac n p \in \mathbb P\). Therefore, the theorem holds.

\begin{proof}
Assume \(p > \sqrt[3] n\) is the smallest prime factor & For conditional \\
\(\frac 1 p < \frac 1 {\sqrt[3] n}\) & Property of inequality \\
\(\frac n p < \frac n {\sqrt[3] n}\) & Property of inequality \\
\(\frac n p < n^{2/3}\) & Property of inequality \\
\(n \neq p\) & A prime cannot equal a composite \\
\(\frac n p \neq 1\) & Algbera \\
Assume \(\frac n p \notin \mathbb P\) & For contradiction\\
\(\frac n p \in \mathbb{P} \leftrightarrow \neg \exists p (p \in \mathbb{P} \wedge 1 < p \leq \sqrt{\frac n p} \wedge p \mid \frac n p)\) & Theorem 2.3 \\
\(\frac n p \notin \mathbb{P} \leftrightarrow \exists j (j \in \mathbb{P} \wedge 1 < j \leq \sqrt{\frac n p} \wedge j \mid \frac n p)\) & Negative biconditional \\
\(\exists j (j \in \mathbb{P} \wedge 1 < j \leq \sqrt{\frac n p} \wedge j \mid \frac n p)\) & Modus Ponens (on biconditional) \\
\(j \in \mathbb{P} \wedge 1 < j \leq \sqrt{\frac n p} \wedge j \mid \frac n p\) & Existential instantiation \\
\(j \leq \sqrt{\frac n p}\) & Simplification \\
\(j \leq n^{1/3}\) & Algebra \\
\(j \mid \frac n p\) & Simplification \\
\(j \mid p \frac n p\) & Theorem 1.3 \\
\(j \mid n\) & Algebra \\
\(j \leq p \wedge j \mid n \wedge j \in \mathbb P\) & Conjunction \\
Contradicts premise for \(p\) \\
Since \(j\) is a smaller prime factor \\
\(\frac n p \in \mathbb P\) & Contradiction
%\(p > \sqrt [3] n \rightarrow \frac n p \in \mathbb P\) & Conditional proof
\end{proof}

Consider the case where \(n \in \mathbb{P}\). The smallest prime factor \(p\) is itself \(n\). \(\frac n p = 1\) (since both non-zero by premises). Therefore, theorem holds.

Consider the case where \(n = 1\). Actually, don't consider the case where \(n = 1\). There are no prime factors of one, so the antecedant is false. The theorem holds.

The theorem holds when \(n \in \cancel {\mathbb P}\), \(n \in \mathbb P\), and \(n = 1\). These are the only three possibilities for positive integers.

\item \textbf{Algorithm: } Input \(n \in \mathbb N \wedge n > 11\). Output two numbers \(a \in \cancel {\mathbb P}\), and \(b \in \cancel {\mathbb P}\) where \(a + b = n\)

\begin{proof}
If \(n \equiv 0 \pmod 2\) \\
\(2 \mid n\) & Definition of modulo \\
\(n > 11\) & Premise \\
\(n - 4 > 7\) & Property of inequalities \\
\(n - 4 \neq 2\) & Property of inequalities \\
\(2|(n - 4)\) & Theorem 1.2 \\
\((n - 4) \in \cancel {\mathbb P}\) & Definition of composite \\
\(4 = 2 \cdot 2\) & \\
\(\exists j (1 < j < 4 \wedge j | 4)\) & Existential generalization \\
\(4 \in \cancel {\mathbb P}\) & Definition of composite \\
\((n - 4) + 4 = n\) & Algebra \\
Output \((n - 4)\) and \(4\) \\
Otherwise \(n \not \equiv 0 \pmod 2\) \\
\(\forall i (i \equiv 0 \pmod 2 \vee i \equiv 1 \pmod 2\) & Lemma \\
\(n \equiv 0 \pmod 2 \vee n \equiv 1 \pmod 2\) & Universal instantiation \\
\(n \not \equiv 0 \pmod 2 \rightarrow n \equiv 1 \pmod 2\)
\(n \equiv 1 \pmod 2\) & Modus ponens \\
\(n > 11\) & Premise \\
\(n - 9 > 9\) & Premise \\
\(2 \mid (9 - 1)\) & \\
\(9 \equiv 1 \pmod 2 \) & Definition of modulo \\
\(n - 9 \equiv 1 - 1 \equiv 0 \pmod 2\) & Theorem 1.13 \\
\(2 \mid (n - 9)\) & Definition of modulo \\
\(n \neq 11\) & Simplification \\
\(n - 9 \neq 2\) & Property of inequality \\
\(n - 9 \in \cancel {\mathbb P}\) & Definition of composite \\
\(9 = 3 \cdot 3\) & \\
\(\exists j (1 < j < 9 \wedge j | 9)\) & Existential generalization \\
\(9 \in \cancel {\mathbb P}\) & Definition of composite \\
\((n - 9) + 9 = n\) & Algebra \\
Output \(n - 9\) and \(9\)
\end{proof}

\item 


\item \textbf{Theorem:} Let \(n\) be a natural number greater than one. \(\exists p \in \mathbb{P} (n \geq p \geq n! + 1)\).

% \(n! > 1\) & Lemma
% \(m|n!\) & Definition of \(n!\)

\begin{proof}
Let \(m \in \mathbb{N} \wedge 1 < m < n\) & (For universal generalization) \\
\(n > 1\) & Premise \\
\(n! \geq 1\) & Fact \\
\(\forall n \in \mathbb{N} (\gcd(n, n + 1) = 1)\) & Theorem 2.32 \\
\(\gcd(n!, n!+1) = 1\) & Universal instantiation \\
Assume \(m|(n!+1)\) & For contradiction\\
\(m \mid n!\) & Definition of factorial \\
& since \(m < n\) \\
\(\gcd(n, n+1) \geq m\) & Definition of gcd \\
\(\gcd(n, n+1) \geq m > 1\) & Restatment \\
& (Premise for \(m\)) \\
\(\gcd(n, n+1) > 1\) & Property of inequalities \\
Therefore \(m \nmid n! + 1\) & Lemma \\
\(\forall m (1 < m < n \rightarrow m \nmid (n!+1))\) & Universal generalization \\
\(\exists p (p \in \mathbb P \wedge p \mid (n!+1))\) & Fundamental Theorem of Arithmetic \\
Let \(p \in \mathbb P \wedge p \mid (n! + 1)\) & Existential instantiation \\
\(1 < p < n \rightarrow p \nmid (n! + 1)\) & Universal instantiation \\
\(p \mid (n!+1) \rightarrow \neg (1 < p < n)\) & Contrapositive \\
\(p \mid (n! + 1)\) & Simplification \\
\(\neg(1 < p < n)\) & Modus ponens \\
\(\neg(1 < p \wedge p < n)\) & Modus ponens \\
\(\neg(1 < p) \vee \neg(p < n)\) & Modus ponens \\
\(\neg(1 < p) \vee p \geq n\) & Property of inequality \\
\(\neg \neg(1 < p) \rightarrow p \geq n\) & Conditional disjunction \\
\(1 < p \rightarrow p \geq n\) & Double negation \\
\(1 < p\) & \(p\) is prime \\
\(p \geq n\) & Modus ponens \\
\(ap = (n! + 1) \fs a \in \mathbb N\) & Definition of divides \\
& and existential instantiation (on \(a\)) \\
\(p \leq (n! + 1)\) & Property of inequality \\
\(n \leq p \wedge p \leq (n! + 1)\) & Conjunction \\
\(n \leq p \leq (n! + 1)\) & Property of inequalities \\
\(\exists p \in \mathbb P (n \leq p \leq (n! + 1))\) & Existential quantification
\end{proof}

\item 

\end{enumerate}

\end{document}

%%% Local Variables:
%%% mode: latex
%%% TeX-master: t
%%% End:
