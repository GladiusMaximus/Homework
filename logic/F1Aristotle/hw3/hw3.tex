\documentclass[12pt,letterpaper]{article}

\usepackage{amssymb}
\usepackage{amsmath}
\usepackage[letterpaper, margin=0.75in]{geometry}
\usepackage{array}
\usepackage{amstext}
\usepackage{setspace}

%\renewcommand{\line}{[-1.5ex]\rule{\linewidth}{.4pt}} % Horizontal line
\newcommand{\lif}{\rightarrow}
\newcommand{\liff}{\leftrightarrow}
\newenvironment{argument}{\begin{tabular}[t]{>{$}l<{$}}}{\end{tabular}}
\newenvironment{myproof}{\begin{tabular}[t]{c >{$}c<{$} c}}{\end{tabular}}

\title{Submission 3.1 (with \LaTeX)}
\date{2014$-$10$-$27}
\author{Sam Grayson}

\begin{document}
\maketitle
\singlespacing

\begin{enumerate}
\item {$Lxy$ is not a function. One person can love two people. So $\exists x \exists y \exists z (Lxy \land Lxz \land y \neq z$}

\item {$Mxy$ is a function mapping from people to people. Each person has exactly one mother, so we know $Mxy$ is a function}

\item {$Cxy$ is not a function. I am the child of Mark and I am the child of Terri, therefore $\exists x \exists y \exists z (Cxy \land Cxz \land y \neq z$}

\item {$Hxy$ is not a function. Multiple people can share the same height.}

\item {$Hxy$ is a function from people to lengths. Each person has exactly one height therefore we can say $Hxy$ is a function.}

\item {$Fxy$ is a function from married men to women. Each married person has exactly one first wife.}

\item {
\begin{myproof}
1.&\forall x f(x, 0) = x&Premise \\
2.&\forall x f(x, g(x)) = 0&Premise \\
3.&\forall x \forall y f(x, y) = f(y, x)&Premise \\
4.&f(g(0), 0) = g(0)&UI(1) \\
5.&f(0, g(0)) = 0&UI(2) \\
6.&f(0, g(0)) = f(g(0), 0)&UI(3) \\
7.&g(0) = 0&Transitive Property (4,6,5) \\ % f(g(0), 0) = g(0) = f(0, g(0)) = 0
&&(Used thrice) \\
8.&\exists g(x) = x&EG(7) \\

\end{myproof}
}

\item {
\begin{myproof}
1.&\forall x f(x, 0) = x&Premise \\
2.&\forall x f(x, g(x)) = 0&Premise \\
3.&f(0, 0) = 0&UI(1) \\
4.&\exists x f(x, x) = x&EG(3) \\
\end{myproof}
}

\item {Neither surjective nor injective.}

\item {Not surjective but injective.}

\item {Both surjective and injective.}

\item {Surjective but not injective.}

\item {Both Surjective and injective.

Every natural number has a representation in Roman Numerals. Therefore $f$ is surjective.

No two natural numbers have the same representation in Roman Numerals. Therefore $f$ is injective.
}

\item {Surjective but not injective

Every element in $B$ is pointed to by at at least one element in $A$. Therefore $f$ is surjective.

$f(1) = a$ and $f(27) = a$, but $1 \neq 27$. Also the domain is larger than the image, so every element can not map to a unique element. Therefore $f$ is not injective.
}

\item {Not surjective but injective.

$\neg \exists x (f(x) = 1 \land x \in Dom(f))$. Therefore $f$ is not surjective.

If $f(x_1) = y = f(x_2)$, then $2 x_1 = 2 x_2$ so $x_1 = x_2$. Therfore $f$ is injective.
}

\item {Surjective and injective.

Given any $y \in Cod(f)$, $f(y - 1) = y$, therefore $\forall y \exists x (y \in Cod(f) \lif (f(x) = y \land x \in Dom(f)))$. Therefore $f$ is surjective.

If $f(x_1) = y = f(x_2)$, then $x_1 - 1 = x_2 - 1$ so $x_1 = x_2$. Therfore $f$ is injective.
}

\item {Neither surjective nor injective.

$\neg \exists x (f(x) = -1 \land x \in Dom(f))$. Therefore $f$ is not surjective.

$f(-2) = 4 = f(2)$, but $2 \neq -2$. Therefore $f$ is not injective.
}

\item {Surjective and injective.

Given any $y \in Cod(f)$, $f(\sqrt[3]{y}) = y$. Therefore $\forall y \exists x (y \in Cod(f) \lif (f(x) = y \land x \in Dom (f)))$. Therefore $f$ is surjective.

If $f(x_1) = y = f(x_2)$, then $x_1^3 = x_2^3 $ so $x_1 = x_2$. Therfore $f$ is injective.
}

\item {Not surjective but injective

$\neg \exists x (f(x) = -1 \land x \in Dom(f))$. Therefore $f$ is not surjective.

$f(x_1) = f(x_2) \lif e^{x_1} = e^{x_2} \lif x_1 = x_2$. Therefore $f$ is injective.
}

\item {Neither surjective nor injective.

$\neg \exists x (f(x) = 2 \land x \in Dom(f))$. Therefore $f$ is not surjective.


$ f(0) = 0 = f(2 \pi) $. Therefore $f$ is not injective.
}

\item {Surjective but not injective.

Given any $y \in Cod(f)$, $f(\sin^{-1} y) = y$. Therefore $\forall y \exists x (y \in Dom(f) \lif (f(x) = y \land x \in Dom(f)))$. Therefore $f$ is surjective.

$ f(0) = 0 = f(2 \pi) $. Therefore $f$ is not injective.
}

\item {Neither surjective nor injective.

Given any $y \in Cod(f)$, $f(\sin^{-1} y) = y$. Therefore $\forall y \exists x (y \in Cod(f) \lif (f(x) = y \land x \in Dom(f)))$. Therefore $f$ is surjective.

$ f(0) = 0 = f(2 \pi) $. Therefore $f$ is not injective.
}

\item {Surjective but not injective.

Given any $y \in Cod(f)$, $f(y) = y$. Therefore $\forall y \exists x (y \in Cod(f) \lif (f(x) = y \land x \in Dom(f)))$. Therefore $f$ is surjective.

$f(\frac{1}{2}) = 0 = f(\frac{1}{3})$, but $\frac{1}{2} \neq \frac{1}{3}$.
}

\end{enumerate}

\end{document}

%%% Local Variables:  
%%% mode: latex 
%%% TeX-master: t 
%%% End: 